\documentclass[a4paper,10pt]{article}
\usepackage[cp1250]{inputenc}           % �umnike lahko vna�amo s tipkovnico
\usepackage{epsfig}
\usepackage[T1]{fontenc}                % kodiranje pisave
\usepackage{eurosym}
\usepackage {algorithmicx}
\usepackage{algpseudocode}
\usepackage{algorithm}
\usepackage{amsfonts}                   % dodatni matemati�ni simboli
\usepackage{amsmath}                        % za sklice na oznake
\usepackage{graphicx}
\usepackage{listings}
\usepackage{fancyvrb}

\bibliographystyle{plain}

%%%%%%%%%%%%%%%%%%%%%%%%%%%%%%%%%%%%%%%%%
\begin{document}
\title{Technical document on ConPaaS services: ConPaaS MySQL Server}
\vspace{15pt}
\author{Ale{\v s} {\v C}ernivec}
\vspace{50pt}
\maketitle
\vspace{15pt}
\setlength{\parindent}{15pt}
\newpage
\tableofcontents
\newpage
%%%%%%%%%%%%%%%%%%%%%%%%%%%%%%%%%%%%%%%%%%%%%%%%%%%%%%%%%%%%%%%%%%%%%
\newcommand{\Cmd}[1]{\noindent {\tt #1 }\newline\vspace{2pt}\\}
\newcommand{\Des}[1]{\vspace{4pt}\noindent {\bf Description: }\\{#1}\vspace{2pt}\\ }
\newcommand{\Par}[1]{\vspace{4pt}\noindent {\bf Parameters: } \\{#1}\vspace{2pt}\\}
\newcommand{\Ret}[1]{\vspace{4pt}\noindent {\bf Returns: }\\{#1}\vspace{2pt}\\}
\newcommand{\Rai}[1]{\vspace{4pt}\noindent {\bf Raises: }\\{#1}\vspace{2pt}\\}
\newcommand{\conapi}[5]{\Cmd{#1} \Des{#2} \Par{#3} \Ret{#4} \Rai{#5}\\}

\section{Introduction}

Currently you can add and remove agent nodes, query for status of the agent nodes, configure users, upload mysql database. 

\section{Architecture}

ConPaaS MySQL Manager node has to be run manually or by the usage of ConPaaS web front-end. ConPaaS Servers can also be managed through direct web front-end (see section \ref{sec:web-front-end}).

This is only for development: when manager starts, it fetches the fresh package of conpaas sources from public location (e.g. \\{\tt http://contrail.xlab.si/conpaassql.tar}). Images will be pre-packaged with mysql manager and agent on the release. Template for creating manager, contains details on installation script (\\{\tt http://contrail.xlab.si/conpaassql/manager/conpaassql-install.sh}, compare to section \ref{sec:install}). When installation is complete, manager can be used to orchestrate ConPaaS SQL agents.

When obtaining access point of the manager, new agents can be provisioned by issuing HTTP POST {\tt add\_nodes} command on resource {\tt sql-manager-host:/}:

\begin{verbatim}
POST / HTTP/1.1
Accept: */*
Content-Type: application/json
Content-Length: 67
{"params": {"function": "agent"}, "method": "add_nodes", "id": "1"}
\end{verbatim}

Parameter {\tt function} will soon support more than just creating new agent nodes (e.g. cluster manager, cluster agent, cluster). Parameter {\tt method} designates command {\tt add\_nodes} and {\tt id } equals 1. 

\section{Configuration}

Configuration of the service consists of two parts:
\begin{itemize}
	\item configuration file of the service describing {\bf iaas section}, {\bf manager section} and {\bf onevm\_agent\_template section}
	\item agent's template file which defines the template used by the manager. The template file is filled in with variables from the configuration file.
\end{itemize}

An example of the configuration file:

\begin{Verbatim}[frame=single]
[iaas]
DRIVER=OPENNEBULA_XMLRPC
OPENNEBULA_URL=$ONE_URL
OPENNEBULA_USER=$ONE_USERNAME
OPENNEBULA_PASSWORD=$ONE_PASSWORD
OPENNEBULA_IMAGE_ID = $AGENT_IMAGE_ID
OPENNEBULA_SIZE_ID = 1
OPENNEBULA_NETWORK_ID = $AGENT_NETWORK_ID
OPENNEBULA_NETWORK_GATEWAY=$IP_GATEWAY
OPENNEBULA_NETWORK_NAMESERVER=$NAMESERVER

[manager]
find_existing_agents = true

[onevm_agent_template]
FILENAME=/root/conpaassql/bin/agent.template
NAME=conpaassql_server
CPU=0.2
MEM_SIZE=256
DISK=bus=scsi,readonly=no
OS=arch=i686,boot = hd, root = hda
IMAGE_ID=\$AGENT_IMAGE_ID
NETWORK_ID=$AGENT_NETWORK_ID
CONTEXT=vmid=\$VMID,vmname=\$NAME,ip_private="$NIC[IP, \
  NETWORK=\$NETWORK"]",ip_gateway="\$IP_GATEWAY",netmask=\
  "\$NETMASK",nameserver="\$NAMESERVER",target=sdc,files=\
  \$AGENT_FILES
\end{Verbatim}

An example of the agent's template file:

\begin{Verbatim}[frame=single]
NAME   = \$NAME
CPU    = \$CPU
MEMORY = \$MEM_SIZE
   OS     = [
   \$OS 
   ]
DISK   = [
   IMAGE_ID = \$IMAGE_ID,
   \$DISK 
   ]
NIC    = [ NETWORK_ID = \$NETWORK_ID ]
GRAPHICS = [
  type="vnc"
  ]
CONTEXT = [
  \$CONTEXT   
  ]
RANK = "- RUNNING_VMS"
\end{Verbatim}

Agent's template file is referenced within the configuration file. This way the contextualisation of the agent is very generic. 

\section{Installation on VM images}
\label{sec:install}

These steps are necessary in order to clean install ConPaaSSQL server on images used on OpenNebula.

First, you will need
\begin{enumerate}
\item {\tt http://contrail.xlab.si/conpaassql/agent/init.sh}
\item {\tt http://contrail.xlab.si/conpaassql/manager/init.sh}
\end{enumerate}

The contents of {\tt init.sh} files is given in appendix \ref{app:install}.

\subsection*{step1}

For deploying ConPaaS SQL Manager image, we are using this template description:

\begin{Verbatim}[frame=single]
NAME = conpaas-sql-manager CPU = 0.5 MEMORY = 128
   OS     = [ boot="hd", arch="x86_64" ]

DISK   = [
   source   = "/home/ales/sql/conpaas-sql-1.qcow2",
   target   = "vda",
   readonly = "no",
   DEV_PREFIX = vd,
   DRIVER = qcow2,
   BUS = virtio   # ide (not IDE), SCSI, virtio
]

NIC    = [ 
   NETWORK = "private-lan"
   ]

GRAPHICS = [
  type="vnc", 
  listen="localhost"
  ]

CONTEXT = [
	ip_private  = "\$NIC[IP, NETWORK=\"private-lan\"]",
    ip_gateway  = "10.1.0.254",
    netmask = "255.255.255.0",
    nameserver = "192.168.122.1",
    agent_files="/home/ales/conpaassql/agent/init.sh",
    agent_image_id="205",
    agent_network_id="205",
    one_username=oneadmin,
    one_password=oneadmin,
    one_url="http://10.30.1.14:2633/RPC2"
  	files = "/home/ales/conpaassql/manager/init.sh",
  	target = "hdb"
]
RANK = "- RUNNING_VMS"
\end{Verbatim}

It is important to carefully fill in the {\bf context} part of the template. There are several properties to fill in:
\begin{itemize}
	\item ip\_private is manager's IP,
	\item ip\_gateway is manager's gateway,
	\item netmask is manager's netmask,		
	\item nameserver is manager's domain name server,			
	\item agent\_files are files which are used in the agent's {\bf init.sh} script,
	\item agent\_image\_id is the agent image's id to be used for provisioning new agents,	
	\item agent\_network\_id is the agent network id to be used for provisioning new agents,		
	\item one\_username is the account used for provisioning new agents,
	\item one\_password is the account used for provisioning new agents,
	\item one\_url is the XMLRPC url to be used by the manager to provision new agents,	
	\item files are the files which appear under the manager as an ISO image,		
	\item target can be left blank (it is for the need for contextualisation).
\end{itemize}

These context variables are after the booting of the manager used by the {\bf manager's init script} (init.sh, see appendix \ref{app:install}) in order to provision new agent's.

\section{ConPaaS MySQL Server Web front-end}
\label{sec:web-front-end}
ConPaaS MySQL Server Web front-end can easily be installed after the server already runs: 

\begin{Verbatim}[frame=single]
# apt-get install python-setuptools python-dev python-pycurl -y
# easy_install pip
# pip install oca apache-libcloud
# pip install --extra-index-url http://eggs.contrail.xlab.si \
conpaassql-manager-gui
# mkdir -p /etc/conpaassql
# cat > /etc/conpaassql/manager-gui.conf << EOF
# MANAGER_HOST = 'localhost'
# EOF
# screen -S conpaasssql-manager-gui -d -m conpaassql_manager_gui
\end{Verbatim}

It also is not mandatory that the web front-end is installed on the same server as the SQL Server already runs. 

\section{ConPaaS MySQL Server Manager API}

Module {\tt conpaas.mysql.server.manager.internals} contains internals of the ConPaaS MySQL Server. ConPaaS MySQL Server consists of several nodes with different roles.

\begin{itemize}
	\item Manager node
	\item Agent node(s)
	\begin{itemize}
		\item Master
		\item Slave(s)
	\end{itemize}	
\end{itemize}

\noindent {\bf platform:}
Linux, Debian Squeeze, tested also within Ubuntu 10.10 (there should be no problem when using later distributions).

\noindent {\bf synopsis:}
Internals of ConPaaS MySQL Servers.

\noindent {\bf moduleauthor:}
Ales Cernivec <ales.cernivec@xlab.si>
      
\vspace{10pt}

\noindent\conapi
{conpaas.mysql.server.manager.internals.add\_nodes(kwargs)}
{HTTP POST method. Creates new node and adds it to the list of existing nodes in the manager. A role of new node can be one of: agent, manager. Currently only agent is supported. It makes internal call to {\tt createServiceNodeThread()}.}
{kwargs -- string describing a function (agent).}
{HttpJsonResponse - JSON response with details about the node.}
{ManagerException}

Example
\begin{Verbatim}[frame=single]
POST / HTTP/1.1
Accept: */*
Content-Type: application/json

Body content: {"params": {"function": "agent"}, "method": /
"add_nodes", "id": "1"}
\end{Verbatim}

\noindent\conapi
{ conpaas.mysql.server.manager.internals.remove\_nodes(params)}
{HTTP POST method. Deletes specific node from a pool of agent nodes. Node deleted is given by {\tt \{'serviceNodeId':id\}.}}
{kwargs --string identifying a node.}
{HttpJsonResponse - HttpJsonResponse - JSON response with details about the node. OK if everything went well. }
{ManagerException if something went wrong. It contains a detailed description about the error.}

Example
\begin{Verbatim}[frame=single]
POST / HTTP/1.1
Accept: */*
Content-Type: application/json

Body content: {"params": {"serviceNodeId": "12"}, "method": /
"remove_nodes", "id": "1"}
\end{Verbatim}

\noindent\conapi
{ conpaas.mysql.server.manager.internals.list\_nodes()}
{ HTTP GET method. Uses {\tt IaaSClient.listVMs()} to get list of all service nodes. For each service node it checks if it is in servers list. If some of them are missing they are removed from the list. Returns list of all service nodes.}
{}
{HttpJsonResponse - JSON response with the list of services: {\tt \{ 'serviceNode': [<a list of ids>]\})}}
{HttpErrorResponse}

Example
\begin{Verbatim}[frame=single]
GET /?method=list_nodes&id=1 HTTP/1.1
Accept: */*
Content-Type: application/json
\end{Verbatim}

\noindent\conapi
{ conpaas.mysql.server.manager.internals.get\_node\_info()}
{HTTP GET method. Gets info of a specific node.}
{param (str) -- serviceNodeId is a VMID of an existing service node.}
{HttpJsonResponse - JSON response with details about the node: : {\tt \{'serviceNode':\{'id': serviceNode.vmid,'ip': serviceNode.ip,'isRunningMySQL': serviceNode.isRunningMySQL\}\}}.}
{ManagerException}

Example (note the VMID inside the command)
\begin{Verbatim}[frame=single]
GET /?params=%7B%22serviceNodeId%22%3A+%22<VMID>%22%7D&method=get_node_info&id=1 HTTP/1.1
Accept: */*
Content-Type: application/json
\end{Verbatim}

\noindent\conapi
{ conpaas.mysql.server.manager.internals.get\_service\_info()}
{HTTP GET method. Returns the current state of the manager.}
{param (str) -- serviceNodeId is a VMID of an existing service node.}
{HttpJsonResponse - JSON response with the description of the state.}
{ManagerException}

Example
\begin{Verbatim}[frame=single]
GET /?method=get_service_info&id=1 HTTP/1.1
Accept: */*
Content-Type: application/json
\end{Verbatim}

\noindent\conapi
{ conpaas.mysql.server.manager.internals.set\_up\_replica\_master()}
{HTTP POST method. Sets up a replica master node.}
{ id -- new replica master id.}
{HttpJsonResponse - JSON response with details about the new
      node. ManagerException if something went wrong.}
{ManagerException}
     
\noindent\conapi
{ conpaas.mysql.server.manager.internals.set\_up\_replica\_slave()}
{HTTP POST method. Sets up a replica master node.}
{  id -- new replica slave id.}
{ HttpJsonResponse - JSON response with details about the new
      node. ManagerException if something went wrong.}
{ManagerException}


\noindent\conapi
{ conpaas.mysql.server.manager.internals.shutdown()}
{HTTP POST method. Shuts down the manager service.}
{  id -- new replica slave id.}
{ HttpJsonResponse - JSON response with details about the status
      of a manager node: :py:attr`S\_EPILOGUE`. ManagerException if
      something went wrong.}
{ManagerException}

\noindent\conapi
{ conpaas.mysql.server.manager.internals.get\_service\_performance()}
{ HTTP GET method. Placeholder for obtaining performance metrics.}
{ kwargs (dict) -- Additional parameters.}
{HttpJsonResponse -- returns metrics}
{}

Example
\begin{Verbatim}[frame=single]
GET /?method=get_service_performance&id=1 HTTP/1.1
Accept: */*
Content-Type: application/json
\end{Verbatim}

\newpage

\section{Conclusion}

\newpage

\section{Appendix A}
\label{app:install}

In this appendix the content of manager's {\tt init.sh} is given.

\begin{Verbatim}[frame=single]
#!/bin/bash
 
if [ -f /mnt/context.sh ]; then
  . /mnt/context.sh
fi

ifconfig eth0 \$IP_PRIVATE netmask \$NETMASK
route add default gw \$IP_GATEWAY eth0
echo nameserver \$NAMESERVER > /etc/resolv.conf

# installation
apt-get update
apt-get install -y unzip python python-mysqldb python-pycurl \
 python-dev python-setuptools python-pip subversion

cd /root
svn co svn://svn.forge.objectweb.org/svnroot/contrail/trunk/\
 conpaas/sql/conpaassql
cd conpaassql
python setup.py bdist_egg
easy_install dist/conpaas*

cat > /root/conpaassql/src/conpaas/mysql/server/manager/configuration.cnf << EOF
[iaas]
DRIVER=OPENNEBULA_XMLRPC
OPENNEBULA_URL=$ONE_URL
OPENNEBULA_USER=$ONE_USERNAME
OPENNEBULA_PASSWORD=$ONE_PASSWORD
OPENNEBULA_IMAGE_ID = $AGENT_IMAGE_ID
OPENNEBULA_SIZE_ID = 1
OPENNEBULA_NETWORK_ID = $AGENT_NETWORK_ID
OPENNEBULA_NETWORK_GATEWAY=$IP_GATEWAY
OPENNEBULA_NETWORK_NAMESERVER=$NAMESERVER

[manager]
find_existing_agents = true

[onevm_agent_template]
FILENAME=/root/conpaassql/bin/agent.template
NAME=conpaassql_server
CPU=0.2
MEM_SIZE=256
DISK=bus=scsi,readonly=no
OS=arch=i686,boot = hd, root = hda
IMAGE_ID=$AGENT_IMAGE_ID
NETWORK_ID=$AGENT_NETWORK_ID
CONTEXT=vmid=$VMID,vmname=$NAME,ip_private="$NIC[IP, NETWORK=$NETWORK"]",\
 ip_gateway="$IP_GATEWAY",netmask="$NETMASK",nameserver="$NAMESERVER",target=sdc,files=$AGENT_FILES
EOF

nohup python /root/conpaassql/src/conpaas/mysql/server/manager/server.py -c \
 /root/conpaassql/src/conpaas/mysql/server/manager/configuration.cnf 1>  \
 /var/log/conpaassql-stdout.log 2> /var/log/conpaassql-err.log &

\end{Verbatim}

The content of agent's {\tt init.sh} is given here:

\begin{Verbatim}[frame=single]
#!/bin/bash

if [ -f /mnt/context.sh ]; then
  . /mnt/context.sh
fi

ifconfig eth0 $IP_PRIVATE netmask $NETMASK
route add default gw $IP_GATEWAY eth0
echo nameserver $NAMESERVER > /etc/resolv.conf

# installation
export DEBIAN_FRONTEND=noninteractive

apt-get update
apt-get install -y unzip python python-mysqldb python-pycurl python-dev \
 python-setuptools python-pip subversion mysql-server supervisor

cd /root
svn co svn://svn.forge.objectweb.org/svnroot/contrail/trunk/conpaas/sql/conpaassql
cd conpaassql
python setup.py bdist_egg
easy_install dist/conpaas*

cat >  /etc/supervisor/conf.d/mysql.conf << EOF
[program:mysqld]
command=/usr/sbin/mysqld
autostart=true
autorestart=true
EOF

cat >>  /etc/supervisor/supervisord.conf << EOF
[inet_http_server]
port = 0.0.0.0:9001
username = root
password = root
EOF

cat >  /root/conpaassql/src/conpaas/mysql/server/agent/configuration.cnf << EOF
[MySQL_root_connection]
location=
password=contrail
username=root

[MySQL_configuration]
my_cnf_file=/etc/mysql/my.cnf

[supervisor]
user = root
password = root
port = 9001

[ConPaaSSQL]
agent_interface=0.0.0.0
agent_port=60000
manager_ip=0.0.0.0
manager_port=50000
vm_id=10
vm_name=agent10
EOF

# Run the agent command in the background

service supervisor stop
service supervisor start

nohup python /root/conpaassql/src/conpaas/mysql/server/agent/server.py -c \
 /root/conpaassql/src/conpaas/mysql/server/agent/configuration.cnf 1> \
 /var/log/conpaassql-stdout.log 2> /var/log/conpaassql-err.log &

\end{Verbatim}

\newpage

\section{Appendix B}
\label{app:init}

In this appendix the content of contextualization of the manager and agent nodes' {\tt init.sh} script is given. First, the manager's {\tt init.sh} script:

\begin{Verbatim}[frame=single]
#!/bin/bash
 
if [ -f /mnt/context.sh ]; then
  . /mnt/context.sh
fi
 
cat > /home/contrail/sql/src/conpaas/mysql/server/manager/configuration.cnf << EOF
[iaas]
DRIVER=OPENNEBULA_XMLRPC
OPENNEBULA_URL=http://10.30.1.1:2633/RPC2
OPENNEBULA_USER=oneadmin
OPENNEBULA_PASSWORD=oneadmin
OPENNEBULA_IMAGE_ID=/home/contrail/sql/conpaas-sql-1.qcow2
OPENNEBULA_NETWORK_ID=205
OPENNEBULA_SIZE_ID=1
OPENNEBULA_CONTEXT_SCRIPT_MANAGER=/home/contrail/manager/conpaassql-install.sh
OPENNEBULA_CONTEXT_SCRIPT_AGENT=/home/contrail/sql/agent/init.sh
EOF

# Run the manager command in the background
DEST_DIR=/home/contrail/sql
cd ${DEST_DIR}/src
PYTHONPATH=${DEST_DIR}/src:${DEST_DIR}/contrib nohup \
 python conpaas/mysql/server/manager/server.py -c\
 ${DEST_DIR}/src/conpaas/mysql/server/manager/configuration.cnf &

\end{Verbatim}

\newpage
\vspace{4mm}

The agent's {\tt init.sh} script:

\begin{Verbatim}[frame=single]
#!/bin/bash
 
if [ -f /mnt/context.sh ]; then
  . /mnt/context.sh
fi
 
cat >  /home/contrail/sql/src/conpaas/mysql/server/agent/configuration.cnf << EOF
[MySQL_root_connection]
location=
password=contrail
username=root

[MySQL_configuration]
my_cnf_file=/etc/mysql/my.cnf
path_mysql_ssr=/etc/init.d/mysql
EOF

# Run the agent command in the background

DEST_DIR=/home/contrail/sql
cd ${DEST_DIR}/src
PYTHONPATH=${DEST_DIR}/src:${DEST_DIR}/contrib nohup \
 python conpaas/mysql/server/agent/server.py -c\
 ${DEST_DIR}/src/conpaas/mysql/server/agent/configuration.cnf &

\end{Verbatim}

\newpage
\vspace{4mm}

\begin{thebibliography}{99}

%	\bibitem{seriesX}Series X: Data Networks, Open System Communications and
%Security, X.509, 08/2005
%	\bibitem{rfcCertAndCRL}	R. Housley et al, Internet X.509 Public Key
%Infrastructure Certificate and Certificate Revocation List (CRL) Profile, RSA
%Laboratories, April 2002
%	\bibitem{federal_office}The Office of the Federal Privacy Commissioner,
%Privacy and Public Key Infrastructure: Guidelines for Agencies using PKI to
%communicate or transact with individuals, 21 December 2001
%	\bibitem {ten_risks} Carl Ellison and Bruce Schneier, Ten Risks of PKI:
%What You are not Being Told about Public Key Infrastructure, Computer Security
%Journal, Volume XVI, Number 1, 2000
%	\bibitem {rfcOCSP} X.509 Internet Public Key Infrastructure Online
%Certificate Status Protocol - OCSP http://www.ietf.org/rfc/rfc2560.txt	
%	\bibitem {applied}Alfred J. Menezes, Paul C. Van Oorschot, Scott A.
%Vanstone, Handbook of Applied Cryptography, 5th ED
%	\bibitem{beginning}Hook D., Beginning Cryptography with Java, 2005
%	\bibitem{securityplus}Pastore M., Dulaney M., Security+, Second
%Edition, Exam SYO-101, 2004
\end{thebibliography}

\end{document}