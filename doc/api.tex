\documentclass[10pt]{article}

\usepackage{hyperlatex} % for HTML generation

\begin{ifhtml}
  
  \begin{ifclear}{noframe}
    \usepackage[simple]{frames}     % for building HTML frames
  \end{ifclear}

  \htmlcss{conpaas.css}
  \providecommand{\url}[1]{\xlink{#1}{#1}}
  \setcounter{htmlautomenu}{4}
  \setcounter{secnumdepth}{2} 
  \setcounter{htmldepth}{1}
  
  \htmladdress{Copyright \copyright 2011-2012 \xlink{Contrail}{http://www.contrail-project.eu} consortium.\\
    All rights reserved.}

  \renewcommand{\maketitle}{\EmptyP{\HlxTitleP}{
      \HlxBlk\xml{div style="width: 78%;"}\xml{frontmatter}
      \xml{h1 align=center}\HlxTitle\xml{/h2}
      \EmptyP{\HlxAuthorP}{\xml{h2 align=center}\HlxAuthor\xml{/h2}}{}
      \EmptyP{\HlxDate}{\xml{h2 align=center}\HlxDate\xml{/h2}}{}
      \xml{/frontmatter}
    }{}}

  \renewcommand{\HlxFramesDescription}[2]{
    \xml{frameset rows="100%" cols="25%,75%"}
      \xml{frame src="#1_toc#2" name="contents" marginwidth="5"
        marginheight="5" border="0"}
      \xml{frame src="#1_0#2" name="main" marginwidth="20" 
        marginheight="20"}
      \xml{noframes}
      This document uses frames to assist navigation.
      Your browser is currently not supporting the use of frames, but you 
      may still access the 
      \xml{a target="_top" href="#1_0#2"}non-framed version\xml{/a}.
      \xml{/noframes}
      \xml{/frameset}}
    
    \renewcommand{\HlxFramesNavigation}{%
      \HlxTocName
      \htmlpanel{0}
      \HlxSection{-5}{}*{\navigationname}
      \xml{base target="main"}
      \xml{div style="width: 26%;"}
      \htmlmenu[0]{2}
      \renewcommand{\bottommatter}{}}

\end{ifhtml}


\T\usepackage{hyperref}
\T\usepackage{url}


\title{ConPaaS -- API documentation\htmlonly{ [\xml{a href="api.pdf"}pdf\xml{/a}]}}
\htmltitle{ConPaaS -- API documentation}
\author{Emanuele Rocca}
\date{ConPaaS-1.1.0}

\begin{document}

\maketitle

%\T\vfil
%\T\tableofcontents
%\T\vfil
%\T\newpage

\section{Introduction}
\label{intro}
ConPaaS services are composed by a manager and one or more agents: the
manager's role is to oversee the functioning of a specific service. ConPaaS
managers are responsible, among other things, for starting up and shutting down
ConPaaS agents, which in turn provide the functionality offered by a specific
service.

ConPaaS services are created and administered exclusively through an HTTP /
JSON Application Programing Interface exposed by a web service called ConPaaS
Director. This document provides a description of such an API.

\pagebreak 

\section{Available methods}
Let us list the ConPaaS API methods, together with a brief description of their
behavior.

\begin{verbatim}
POST /new_user

    Create a new ConPaaS user. The method expects the following parameters:

    'username', 'fname', 'lname', 'email', 'affiliation', 'password', 'credit'
    
    A dictionary of user values is returned upon successful user creation.
    The following dictionary is returned on failure:

    {
     'error': True,
     'msg': 'An explainatory error message' 
    }

POST /login

    Authenticates the given user. The following parameters are expected:
    'username', 'password'

    A dictionary of user values is returned upon successful authentication.
    False is returned otherwise.

POST /get_user_certs

    Create and send SSL certificates for the given user. 'username' and
    'password' are the expected parameters. A zip archive containing the SSL
    certificates is returned on success, False otherwise.
    
GET /available_services

    Return a list of available service types. For example: 
    ['scalaris', 'selenium', 'hadoop', 'mysql', 'java', 'php']

POST /start/<servicetype>

    Return a dictionary with service data (manager's vmid and IP address,
    service name and ID) in case of correct service creation. False is returned
    otherwise. Only service types returned by the 'available_services' method
    described above are allowed. 

    This method requires the client to present a valid SSL certificate.

POST /stop/<serviceid>

    Return a boolean value. True in case of proper service termination, false
    otherwise. <serviceid> has to be an integer representing the service id of a
    running service.

    This method requires the client to present a valid SSL certificate.

POST /rename/<serviceid>

    Rename the given service. The new name 'name' is the only required
    argument. Return true on successful renaming, false otherwise.

GET  /list

    List running ConPaaS services. Return data as a list of dictionaries
    (associative arrays).

    This method requires the client to present a valid SSL certificate.

GET  /download/ConPaaS.tar.gz

    Used by ConPaaS services. Download a tarball with the ConPaaS source code.

POST /callback/decrementUserCredit.php
    
    Used by ConPaaS services. 'sid' and 'decrement' are the required parameters.

    Decrement user credit and check if it is enough.  Return a dictionary with
    the 'error' attribute set to false if the user had enough credit, true
    otherwise.

    This method requires the client to present a valid SSL certificate of type
    'manager'. The 'serviceLocator' field in the supplied certificate has to match
    the 'sid'.

POST /ca/get_cert.php

    Used by ConPaaS services. The only required argument is a file called 'csr'
    holding a certificate signing request. A certificate is returned.

    This method requires the client to present a valid SSL certificate of type
    'manager'..
\end{verbatim}

The first three methods, namely \textbf{new\_user}, \textbf{login} and
\textbf{get\_user\_certs} do not need a client SSL certificate to be called.

\textbf{available\_services}, \textbf{start}, \textbf{stop}, \textbf{rename}
and \textbf{list} all need a valid user certificate in order to be called.

The last two methods are used by ConPaaS managers to decrement users' credit
and create agent certificates. They both need a valid manager certificate to be
called.
\end{document}
