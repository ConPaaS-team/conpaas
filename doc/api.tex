\documentclass[10pt]{article}

\usepackage{hyperlatex} % for HTML generation

\begin{ifhtml}
  
  \begin{ifclear}{noframe}
    \usepackage[simple]{frames}     % for building HTML frames
  \end{ifclear}

  \htmlcss{conpaas.css}
  \providecommand{\url}[1]{\xlink{#1}{#1}}
  \setcounter{htmlautomenu}{4}
  \setcounter{secnumdepth}{2} 
  \setcounter{htmldepth}{1}
  
  \htmladdress{Copyright \copyright 2011-2012 \xlink{Contrail}{http://www.contrail-project.eu} consortium.\\
    All rights reserved.}

  \renewcommand{\maketitle}{\EmptyP{\HlxTitleP}{
      \HlxBlk\xml{div style="width: 78%;"}\xml{frontmatter}
      \xml{h1 align=center}\HlxTitle\xml{/h2}
      \EmptyP{\HlxAuthorP}{\xml{h2 align=center}\HlxAuthor\xml{/h2}}{}
      \EmptyP{\HlxDate}{\xml{h2 align=center}\HlxDate\xml{/h2}}{}
      \xml{/frontmatter}
    }{}}

  \renewcommand{\HlxFramesDescription}[2]{
    \xml{frameset rows="100%" cols="25%,75%"}
      \xml{frame src="#1_toc#2" name="contents" marginwidth="5"
        marginheight="5" border="0"}
      \xml{frame src="#1_0#2" name="main" marginwidth="20" 
        marginheight="20"}
      \xml{noframes}
      This document uses frames to assist navigation.
      Your browser is currently not supporting the use of frames, but you 
      may still access the 
      \xml{a target="_top" href="#1_0#2"}non-framed version\xml{/a}.
      \xml{/noframes}
      \xml{/frameset}}
    
    \renewcommand{\HlxFramesNavigation}{%
      \HlxTocName
      \htmlpanel{0}
      \HlxSection{-5}{}*{\navigationname}
      \xml{base target="main"}
      \xml{div style="width: 26%;"}
      \htmlmenu[0]{2}
      \renewcommand{\bottommatter}{}}

\end{ifhtml}


\T\usepackage{hyperref}
\T\usepackage{url}


\title{ConPaaS -- API documentation\htmlonly{ [\xml{a href="api.pdf"}pdf\xml{/a}]}}
\htmltitle{ConPaaS -- API documentation}
\author{Emanuele Rocca}
\date{ConPaaS-1.0.0}

\begin{document}

\maketitle

%\T\vfil
%\T\tableofcontents
%\T\vfil
%\T\newpage

\section{Introduction}
\label{intro}
ConPaaS services are composed by a manager and one or more agents: the
manager's role is to oversee the functioning of a specific service. ConPaaS
managers are responsible, among other things, for starting up and shutting down
ConPaaS agents, which in turn provide the functionality offered by a specific
service.

It is possible to create and administer ConPaaS services through an HTTP/JSON
Application Programing Interface exposed by a web service called ConPaaS
Director. 

\pagebreak 

\section{Available methods}
Let us list the ConPaaS API methods, together with a brief description of their
behavior.

\begin{verbatim}
GET /available_services

    Return a list of available service types. For example: 
    ['scalaris', 'selenium', 'hadoop', 'mysql', 'java', 'php']

POST /start/<servicetype>

    Return a dictionary with service data (manager's vmid and IP address,
    service name and ID) in case of correct service creation. False is returned
    otherwise. Only service types returned by the 'available_services' method
    described above are allowed.

POST /stop/<serviceid>

    Return a boolean value. True in case of proper service termination, false
    otherwise. <serviceid> has to be an integer representing the service id of a
    running service.

GET  /list

    List running ConPaaS services. Return data as a list of dictionaries
    (associative arrays).

GET|POST /manager

    Proxy GET/POST requests to the manager responsible for the given service id
    (the sid parameter). This method allows to call any method made available by
    managers through the director.

GET  /download/ConPaaS.tar.gz

    Used by ConPaaS services. Download a tarball with the ConPaaS source code.

POST /callback/decrementUserCredit.php
    
    Used by ConPaaS services. Decrement user credit and check if it is enough.
    Return a dictionary with the 'error' attribute set to false if the user had
    enough credit, true otherwise.
\end{verbatim}

The first four methods, namely \textbf{start}, \textbf{stop}, \textbf{list} and
\textbf{manager} are of interest for developers wishing to control a ConPaaS
installation from their programs. They all require two additional parameters:
\textbf{username} and \textbf{password}. The remaining methods are, conversely,
only used internally by ConPaaS services and they do not require any additional
parameter. 

One of the above mentioned methods is worth a particular mention: the
\textbf{manager} method can be used to access any method available on the
manager responsible for a specific service. For example, a developer might want
to use the manager method to get the logs of running service, or add and remove
agents. 
\end{document}
