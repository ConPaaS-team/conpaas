\documentclass[10pt]{article}

\usepackage{hyperlatex} % for HTML generation
\htmlcss{conpaas.css}
\providecommand{\url}[1]{\xlink{#1}{#1}}

\title{ConPaaS -- Documentation}
\htmltitle{ConPaaS -- Documentation}
  \htmladdress{Copyright \copyright 2011-2012 \xlink{Contrail}{http://www.contrail-project.eu} consortium.\\
    All rights reserved.}
\date{ConPaaS-1.1.0}

\renewcommand{\maketitle}{\EmptyP{\HlxTitleP}{
    \HlxBlk\xml{frontmatter}
    \xml{h1 align=center}\HlxTitle\xml{/h2}
    \EmptyP{\HlxAuthorP}{\xml{h2 align=center}\HlxAuthor\xml{/h2}}{}
    \EmptyP{\HlxDate}{\xml{h2 align=center}\HlxDate\xml{/h2}}{}
    \xml{/frontmatter}
  }{}}

\begin{document}

\maketitle

The ConPaaS documentation is organized in three parts:

\begin{itemize}
\item \textbf{ConPaaS user guide}:
  [ \xlink{html}{userguide.html} | \xlink{pdf}{userguide.pdf} ]
\item \textbf{ConPaaS installation guide}:
  [ \xlink{html}{installation.html} | \xlink{pdf}{installation.pdf} ]
\item \textbf{ConPaaS internals guide}: [ \xlink{pdf}{internals.pdf} ]
\end{itemize}

\end{document}

