%Why is automatic scaling in the cloud important/useful? (give a real-world example)


%Current possibilities for users who want to run web applications in the
%cloud: 

%1) Systems currently available in clouds: most of them define triggers for 
%adding and removing VMs from the application, based on the monitoring 
%data available in the cloud. Most of these systems are recent and there 
%is a lack of documentation on best practices
%and performance reports of web applications using these systems. 

%2) At the other end of the spectrum: several research works on resource 
%provisioning, proposing more sophisticated techniques based on
%queuing models, prediction and workload analysis. However most of these
%research techniques are tested with synthetic (or partially synthetic)
%benchmarks and it is difficult to estimate how they would perform
%for a real-world application; also not all of them are compatible
%with the current cloud environments.

%Our goal is to bridge the gap between the state-of the art research in
%resource provisioning and the real-world web application deployments.
%In order to achieve this goal:

%\begin{itemize}
%\item we built a Platform-as-a-Service system that can be readily used
%on top of some of today's most popular clouds (Amazon EC2, OpenNebula)
%\item we integrated and adapted in our system some resource  provisioning 
%mechanisms developed in previous research from our group [Jiang]
%\item we tested our system with a Wikipedia deployment and access
%traces  
%\end{itemize}   

%{\em \textbf{Guillaume's version:}
 % \begin{enumerate}
 % \item The cloud is a great place to run Web application. In
  %  particular it opens the door to resource provisioning, since
   % computing resources are available on-demand.
 % \item There are lots of research papers dedicated to sophisticated
  %  techniques to handle resource provisioning. However, if we look at
  %  real deployments we see that they rely on extremely simple
  %  techniques, and completely ignore the results from academic
  %  research on the topic.
 % \item There can be two explanations for this discrepancy: (i) the
 %   gains of using sophisticated techniques are too low for anyone to
 %   bother; (ii) implementing these techniques is a difficult
  %  exercise, which is why real cloud systems rely on simpler
  %  techniques.
 % \item This paper tries to identify the real cause. We do this by
 %   implementing a sophisticated provisioning system in realistic
   % conditions, and reporting on (i) how hard implementation was; and
  %  (ii) potential gains from using the better technique as compared
  %  to a simple strawman.
 % \end{enumerate}
% }

%With the rise of cloud computing, more and more applications started to be deployed over clouds.
% infinite pool of resources


%\lettrine{O}{ne of} the major innovations provided by Cloud computing
%platforms is their pay-per-use model where applications can request
%and release resources at any time according to their needs --- and pay
%only for the resources they actually used. This business model is
%particularly favorable for application domains where workloads vary
%widely over time, such as the domain of Web applications.

\lettrine{O}{ne of} the major innovations provided by Cloud computing
platforms is their pay-per-use model where clients pay only for the
resources they actually use. This business model is particularly
favorable for application domains where workloads vary widely over
time, such as the domain of Web application hosting. Web applications
in the cloud can request and release resources at any time according
to their needs.

However, provisioning the right volume of resources for a Web
application is not a simple task. Web applications are usually
composed of multiple types of components such as Web servers,
application servers, database servers and load balancers that
distribute the incoming traffic across them. Complex performance
behavior of these components makes it difficult to find the optimal
resource allocation, even when the application workload is perfectly
stable. The magnitude of the problem is further increased by the fact
that Web application workloads are often very unstable and hard to
predict. In the case of a sudden load increase there is a necessary
tradeoff between reacting as early as possible to minimize the
duration when the application underperforms because of insufficient
processing capacity, and a slower approach to avoid situations where
the load has already decreased when the new resources become
available.

Faced with this difficult scientific challenge, the academic community
has proposed a wide range of sophisticated resource provisioning
algorithms~\cite{dejun2011,muppala_regression-based_2012, urgaonkar_agile_2008, vasic_dejavu_2012}. 
%However, we observe a wide discrepancy
%between these academic proposals and the very simple techniques
%actually used by the cloud industry where provisioning decisions are
%usually triggered by lower/upper thresholds on resource
%utilization. 
However, we observe a wide discrepancy between these academic
propositions and the very simple mechanisms that are currently
available to cloud customers. These mechanisms are usually based on
lower or upper thresholds on resource utilization. Crossing one of the
thresholds triggers a pre-defined resource provisioning action such as
adding or removing one machine.

We postulate three possible reasons why sophisticated techniques are
not more widely deployed: \emph{(i)} the gains of using
sophisticated provisioning strategies are too low to be worth the
effort; \emph{(ii)} implementing and evaluating these techniques is a
difficult exercise, which is why real cloud systems rely on simpler
techniques; and \emph{(iii)} academic approaches mostly focus on
unrealistic evaluations using simple applications and artificial
workloads~\cite{do_profiling_2011, islam_empirical_2012}. % By using
% real production applications, new
% limitations could be detected and solved in the provisioning systems
% due to the peculiarity and heterogeneity of real workloads. Indeed,
% these workloads have several interesting properties such as a variable
% amount of request volumes, a large diversity of request mix and
% different execution times between requests, which increase the
% difficulties to make scaling decisions.

This paper investigate which of these possible causes are the real
problems, and aims to propose automatic scaling algorithms which
provide better results than the simple ones without overly increasing
complexity.
%This paper aims to identify the real cause of why cloud providers use simpler provisioning techniques.
We implemented several resource provisioning mechanisms in ConPaaS, an
open source platform-as-a-service environment for hosting cloud
applications~\cite{conpaasIC}. Our techniques use several levels of
thresholds to predict future performance degradations, workload trend
detection to better handle traffic spikes and dynamic load balancing
weights to handle resources heterogeneity. We exercised these
algorithms in realistic unstable workload situations by deploying a
copy of Wikipedia and replaying a fraction of the real access traces
to its official web site. Finally, we report on \emph{(i)}
implementation complexity; and \emph{(ii)} potential gains compared to
the threshold-based solution. 
%For the sake of comparison and discussion, we deployed the MediaWiki application and used real access traces to validate our technique, opening doors to real implementation of promising auto-scaling systems.


%Section~\ref{conpaas} introduces the ConPaaS runtime environment. Section~\ref{wikipedia} describes the MediaWiki application, its realistic benchmark, and the peculiarities of the Wikipedia workload-mix utilized to validate our system. Section~\ref{provisioning} focus on the different resource provisioning techniques implemented in ConPaaS. Section~\ref{experiments} details the experimental campaign and its results. Section~\ref{studies} discusses related works. Section~\ref{conclu} draws a conclusion.

