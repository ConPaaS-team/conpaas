
{\bf B. Urgaonkar et al., Agile Dynamic provisioning of Multi-Tier Internet Applications:}
They use predictive and reactive provisioning. They propose a queuing model
for multi-tier applications, but I don't think it's very accurate (they use
a G/G/1 queue, and don't take into account processor sharing at all. At
the end there is a paragraph to address multi-threaded servers, probably
added at the request of a reviewer; but it doesn't match with what's 
in the previous paragraphs). For testing they use Rubis and Rubbos. To generate
workload for Rubis they use some processed access traces from a real-world site.
The interesting part: they did have flash crowds in their traces. 

{\bf Z. Whang et al., AppRAISE:}
Similarly to the previous paper, they use predictive and reactive provisioning.
But they work at the hardware level, adjusting the CPU capacity allocated to
VMs. They also use a queuing model for this. For testing they use Rubis,
but with processed traces from a real-world application.

[Homogeneous experiments] [No real traces]

{\bf H. Zhang et al., Resilient Workload Manager:} 
This paper splits the workload in ``base workload'' and ``tresspassing'' (flash crowd)
workload. These workloads are served by different groups of servers. They attempt to 
divide the data items into popular and less popular, and place them in the right
group of servers. It's not clear how they do the actual provisioning.

[Heterogeneous experiments] [Real Traces] [No profiling]

{\bf N. Kaviani et al., Profiling-as-a-Service:}
The aim of this paper is to provide an application profiling service. To do
this, they substitute at runtime a regular VM from the application with
a VM that has the same application, but with profiling instrumentation.
They don't have much results and also not much details about how the
profiling is done.

[Homogeneous experiments] [Use of profiling techniques] [No Real traces]

{\bf http://www.aschroder.com/2012/01/using-aws-auto-scaling-with-an-elastic-load-balancer-cluster-on-ec2/ }
A blog post with advice on setting up AWS auto-scaling.
