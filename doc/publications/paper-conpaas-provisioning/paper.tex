\documentclass[10pt,twocolumn]{article}

\usepackage{url}
\usepackage{ulem}\normalem
\usepackage{graphicx}
\usepackage{lettrine}
\usepackage{multicol}
\usepackage[usenames,dvipsnames,table]{xcolor}

% \newcommand{\commentary}[2]{{\bf {\sc #1:} \emph{#2}}}
% \newcommand{\guillaume}[1]{\commentary{guillaume}{#1}}
% \newcommand{\corina}[1]{\commentary{corina}{#1}}

\clubpenalty=10000
\widowpenalty=10000

\renewcommand{\topfraction}{0.9}	% max fraction of floats at top
\renewcommand{\textfraction}{0.1}
\renewcommand{\floatpagefraction}{0.8}
\setcounter{topnumber}{1}

\begin{document}

\begin{multicols}{2}
\title{Automatic Scaling for Cloud Web Applications in Practice}
\author{Corina Stratan, Guillaume Pierre, Hector Fernandez, Eliana Tirsa and Valentin Cristea} 
\date{Vrije Universiteit Amsterdam and University Politehnica of Bucharest}
\maketitle
\end{multicols}


Other versions for the title:

A Practical Approach for Automatic Scaling (Resource Provisioning?) of
Web Applications in the Cloud

Implementing Automatic Scaling for Cloud Web Applications


\section*{Introduction}

Why is automatic scaling in the cloud important/useful? (give a real-world example)

Challenges:
\begin{itemize}
\item web applications have multiple tiers with different performance
models and requirements
\item cloud resources are often heterogeneous
\item the behaviour of the servers might change in time, either due to
changes in the type of workload or to VM migration
\item it is difficult to estimate what will be the performance effect
of provisioning and de-provisioning the application with resources
\end{itemize}


The automatic scaling mechanisms currently available in clouds are based
on simple rules and do not take all these issues into account. Our solution
aims to address them as follows:

\begin{itemize}
\item monitoring tier-specific metrics like request rate and response time
\item maintaining a performance profile for each VM instance, that is updated
dynamically
\item a provisioning mechanism that uses the performance profiles to estimate
the potential effect of adding and removing resources
\item load balancing with dynamically adjusted weights
\end{itemize}

Our solution was tested with a real-world application and access traces (Wikibench).


\section*{System overview}

\begin{itemize}
\item performance profiling (offline and online)
\item dynamic weighted load balancing
\item resource provisioning mechanism
\end{itemize}


\section*{Implementation in ConPaaS}


\begin{itemize}
\item ConPaaS overview
\item Ganglia integration
\item the provisioning / profiling manager
\end{itemize}


\section*{The Wikipedia workload}

Describe the application and the workload traces. Explain
the particularities of this application, like:

\begin{itemize}
\item the PHP pages take significantly more time to process
than the web pages
\item each PHP page needs a lot of DB queries 
\item the pages vary in complexity so it is difficult to make
predictions (maybe compute the standard deviation of the response time
from a trace to show this)
\end{itemize}


\section*{Experiments}

1. Basic provisioning (similar to the currently available mechanisms in clouds):
experiment on the DAS-3 nodes


2. Heterogeneity-aware provisioning: experiment on the UPB nodes with both the
basic and heterogeneity-aware provisioning. Compare them in terms of:

\begin{itemize}
\item SLA compliance (for example the percentage of requests that met the SLA?)
\item amount of used resources
\end{itemize}


\section*{Conclusion}


?


\section*{Acknowledgments}

This work is partially funded by the FP7 Programme of the European
Commission in the context of the Contrail project under Grant
Agreement FP7-ICT-257438.

Probably also mention ERRIC here.


\bibliographystyle{plain}
\bibliography{paper}

\end{document}
