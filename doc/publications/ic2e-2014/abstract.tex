Improving resource management of heterogeneous cloud infrastructures is an important research challenge. The wide diversity of cloud-based applications and customers with different QoS requirements have recently exhibited the weaknesses of current provisioning systems. Today's cloud infrastructures provide resource provisioning systems that dynamically adapt the computational power of applications by adding or releasing resources. Unfortunately, these scaling systems present some limitations: \emph{(i)} restrict themselves to a type of resource thus limitating the wide broad of heterogenous clouds; \emph{(ii)} unable to handle workload that fluctuates following an irregular pattern (e.g. sudden traffic spikes); and \emph{(iii)} offer the same QoS service independently of customer preferences such as different levels of service availability and performance. These drawbacks, and customer preferences, are pushing scaling systems toward more adaptative and resource-heterogeneous decision-making algorithms. In this paper, we present an autoscaling system that benefits from the heregeneity of cloud environments to better satisfy the customer requirements even under sudden traffic spikes. The proposed system selects the scaling plan that better fulfills the workload and customer requirements. Experiments conducted on public and private infrastructures establish the benefits when using our system to handle workload variations.

%When handling traffic spikes, we shown how different types of resources combinations can fulfill the SLA without drastically increasing the operationa cost. We developed an adaptative autoscaling system that based on the customer preferences adapt the scaling decision based on.
%Experiments conducted on public and private infrastructures to validate our approach.

%Moreover, the adaptation of the computational power becomes more difficult when hosting applications whose 

%^the workload 
%uctuates following an irregular
%pattern, with occasional short trac spikes.

%these
%traditional benchmarks do not exhibit features of
%real Web applications such as traffic unstability as
%well as heterogeneous and constantly-changing re-
%quest mixes [4].

