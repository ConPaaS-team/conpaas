Improving resource provisioning of heterogeneous cloud infrastructures is an important research challenge. The wide diversity of cloud-based applications and customers with different QoS requirements have recently exhibited the weaknesses of current provisioning systems. Today's cloud infra\-structures provide provisioning systems that dynamically adapt the computational power of applications by adding or releasing resources. Unfortunately, these scaling systems are fairly limited: \emph{(i)} They restrict themselves to a single type of resource; \emph{(ii)} they are unable to fulfill QoS requirements in face of spiky workload; and \emph{(iii)} they offer the same QoS level to all their customers, independent of customer preferences such as different levels of service availability and performance. In this paper, we present an autoscaling system that overcomes these limitations by exploiting heterogeneous types of resources, and by defining multiple levels of QoS requirements. The proposed system selects a resource scaling plan according to both workload and customer requirements. Our experiments conducted on both public and private infrastructures show significant reductions in QoS-level violations when faced with highly variable workloads.
