\documentclass[10pt]{article}

\usepackage{hyperlatex} % for HTML generation

\begin{ifhtml}
  
  \begin{ifclear}{noframe}
    \usepackage[simple]{frames}     % for building HTML frames
  \end{ifclear}

  \htmlcss{conpaas.css}
  \providecommand{\url}[1]{\xlink{#1}{#1}}
  \setcounter{htmlautomenu}{4}
  \setcounter{secnumdepth}{2} 
  \setcounter{htmldepth}{1}
  
  \htmladdress{Copyright \copyright 2011-2012 \xlink{Contrail}{http://www.contrail-project.eu} consortium.\\
    All rights reserved.}

  \renewcommand{\maketitle}{\EmptyP{\HlxTitleP}{
      \HlxBlk\xml{div style="width: 78%;"}\xml{frontmatter}
      \xml{h1 align=center}\HlxTitle\xml{/h2}
      \EmptyP{\HlxAuthorP}{\xml{h2 align=center}\HlxAuthor\xml{/h2}}{}
      \EmptyP{\HlxDate}{\xml{h2 align=center}\HlxDate\xml{/h2}}{}
      \xml{/frontmatter}
    }{}}

  \renewcommand{\HlxFramesDescription}[2]{
    \xml{frameset rows="100%" cols="25%,75%"}
      \xml{frame src="#1_toc#2" name="contents" marginwidth="5"
        marginheight="5" border="0"}
      \xml{frame src="#1_0#2" name="main" marginwidth="20" 
        marginheight="20"}
      \xml{noframes}
      This document uses frames to assist navigation.
      Your browser is currently not supporting the use of frames, but you 
      may still access the 
      \xml{a target="_top" href="#1_0#2"}non-framed version\xml{/a}.
      \xml{/noframes}
      \xml{/frameset}}
    
    \renewcommand{\HlxFramesNavigation}{%
      \HlxTocName
      \htmlpanel{0}
      \HlxSection{-5}{}*{\navigationname}
      \xml{base target="main"}
      \xml{div style="width: 26%;"}
      \htmlmenu[0]{2}
      \renewcommand{\bottommatter}{}}

\end{ifhtml}


\T\usepackage{hyperref}
\T\usepackage{url}


\title{ConPaaS -- Installation guide\htmlonly{ [\xml{a href="installation.pdf"}pdf\xml{/a}]}}
\htmltitle{ConPaaS -- Installation guide}
\author{Ismail El Helw \and Adriana Szekeres \and Guillaume Pierre \and Emanuele Rocca}
\date{ConPaaS-1.1.0}

\begin{document}

\maketitle

\T\vfil
\T\tableofcontents
\T\vfil
\T\newpage

\section{Installation Overview}
\label{sec:overview}

\xlink{ConPaaS}{http://www.conpaas.eu} is composed of two parts: a web service,
called \textbf{cpsdirector}, and a web front-end called \textbf{cpsfrontend}.
The web service implements a restful API which is used by the ConPaaS command
line tools and by the web front-end.

Both the director and the front-end can run inside or outside the cloud.
ConPaaS services are designed to run either in an OpenNebula cloud
installation or in the Amazon Web Services cloud.

Installing ConPaaS requires to take the following steps:

\begin{enumerate}
\item Choose a VM image customized for hosting the services, or create a new
one. Details on how to do this vary depending on the choice of cloud where
ConPaaS will run. Instructions on how to find or create a ConPaaS image
suitable to run on Amazon EC2 can be found in Section~\ref{sec:ec2image}. 
Section~\ref{sec:oneimage} describes how to create a ConPaaS image for OpenNebula.
\item Setup and configure the ConPaaS director. All system
  configuration takes place in the director. Director installation and
  configuration is discussed in Section~\ref{sec:director}.
\item Install the ConPaaS front-end and configure it to use your ConPaaS
  director as explained in Section~\ref{sec:frontend}.
\end{enumerate}

\section{Install and configure the ConPaaS Director}
\label{sec:director}

The ConPaaS Director is a web service that allows users to manage their ConPaaS
services. Users can create, configure and terminate services through it. This
section describes the process of setting up a ConPaaS director on a Debian (or
Ubuntu) system.

\begin{enumerate}
\item Install the required packages: 

\texttt{apt-get update}

\texttt{apt-get install build-essential python-setuptools python-dev libapache2-mod-wsgi libcurl4-openssl-dev}
\item Download \url{http://www.conpaas.eu/dl/cpsdirector-1.1.0.tar.gz} and uncompress it
\item Run "\texttt{make install}" as root
\item Edit \texttt{/etc/cpsdirector/director.cfg} providing your cloud
 configuration. Among other things, you will have to choose an Amazon Machine
 Image (AMI) in case you want to use ConPaaS on Amazon EC2, or an OpenNebula
 image if you want to use ConPaaS on OpenNebula. Section~\ref{sec:ec2image}
 explains how to use the Amazon Machine Images provided by the ConPaaS team, as
 well as how to make your own images if you wish to do so. A description of how
 to create an OpenNebula image suitable for ConPaaS is available in
 Section~\ref{sec:oneimage}.
\end{enumerate}

The installation process will create an Apache VirtualHost for the ConPaaS
director in \texttt{/etc/apache2/sites-available/conpaas-director}. There should
be no need for you to modify such a file, unless its defaults conflict with your
Apache configuration.

Run the following commands as root to start your ConPaaS director for the first
time:

\begin{verbatim}
a2enmod ssl
a2ensite conpaas-director
service apache2 restart
\end{verbatim}

If you experience any problems with the previously mentioned commands, it might
be that the default VirtualHost created by the ConPaaS director installation
process conflicts with your Apache configuration. The Apache Virtual Host
documentation might be useful to fix those issues:
\url{http://httpd.apache.org/docs/2.2/vhosts/}.

\subsection{Important note on SSL certificates}
ConPaaS uses SSL certificates in order to secure the communication between you
and the director, but also to ensure that only authorized parties such as
yourself and the various component of ConPaaS can interact with the system.

It is therefore crucial that the SSL certificate of your director contains the
proper information. In particular, the commonName field of the certificate
should carry the public hostname of your frontend server. The installation
procedure takes care of setting up such a field. However, should your director
hostname change, please ensure you run the following commands:

\begin{verbatim}
sudo cpsconf.py
sudo service apache2 restart
\end{verbatim}

\section{Setup the ConPaaS Frontend}
\label{sec:frontend}

The ConPaaS Frontend can be downloaded from
\url{http://www.conpaas.eu/dl/cpsfrontend-1.1.0.tar.gz}.

After having uncompressed it you should:

\begin{itemize}
\item Install the \texttt{libapache2-mod-php5} and \texttt{php5-curl} Debian packages
\item Copy all the files contained in the \texttt{www} directory underneath your web server document root
\item Copy \texttt{conf/main.ini} and \texttt{conf/welcome.txt} in your ConPaaS
  Director configuration folder (\texttt{/etc/cpsdirector}). Modify those files to suit
  your needs
\item Create a \texttt{config.php} file in the web server directory where you have chosen to install the
  frontend. Please refer to \texttt{config-example.php} for a detailed explaination of
  all the configuration options. Note that \texttt{config.php} must contain the
  \texttt{CONPAAS\_CONF\_DIR} option, pointing to the directory mentioned in the previous
  step
\end{itemize}

Enable SSL if you want to use your frontend via https, for example by issuing
the following commands:

\begin{verbatim}
a2enmod ssl
a2ensite default-ssl
\end{verbatim}

Details about the SSL certificate you want to use have to be specified in
\texttt{/etc/apache2/sites-available/default-ssl}.

As a last step, restart your Apache web server:

\begin{verbatim}
service apache2 restart
\end{verbatim}

At this point, your front-end should be working!

\section{Using ConPaaS on Amazon EC2}
\label{sec:ec2image}

The Web Hosting Service is capable of running over the Elastic Compute
Cloud (EC2) of Amazon Web Services (AWS). This section describes the
process of configuring an AWS account to run the Web Hosting Service.
You can skip this section if you plan to install ConPaaS over
OpenNebula.

If you are new to EC2, you will need to create an account at
\url{http://aws.amazon.com/ec2/}. A very good introduction to EC2 can be found
at \url{http://docs.amazonwebservices.com/AWSEC2/latest/GettingStartedGuide/}.

\subsection{Pre-built EBS Amazon Machine Image}

The Web Hosting Service requires the usage of an Amazon Machine Image (AMI) to
contain the dependencies of its processes. For your convenience we provide a
pre-built public AMI, already configured and ready to be used on Amazon EC2,
for each availability zone supported by ConPaaS. The AMI IDs of said images
are:

\begin{itemize}
    \item \verb+ami-4055db70+ United States West (Oregon)
    \item \verb+ami-4b249322+ United States East (Northern Virginia)
    \item \verb+ami-db0a0ba+ Europe West (Ireland)
\end{itemize}

You can use one of these values when configuring your ConPaaS director
installation as described in Section~\ref{sec:director}.

\subsection{Create an EBS backed AMI on Amazon EC2}

Using pre-built Amazon Machine Images is the recommended way of running ConPaaS
on Amazon EC2, as described in the previous section. However, you can also
create a new Elastic Block Store backed Amazon Machine Image yourself, for example
in case you wish to run ConPaaS in a different Availability Zone.
The easiest way to do that is to start from an already existing AMI, customize
it and save the resulting filesystem as a new image. The following steps explains
how to setup an AMI using this methodology.

\begin{enumerate}
\item Log in the AWS management console, select the ``EC2'' tab, then ``AMIs''
in the left-side menu. Search the public AMIs for a Debian Squeeze EBS AMI and
start an instance of it. If you are going to use micro-instances then the AMI
with ID \verb+ami-e0e11289+ in the US East zone could be a good choice.

\item Upload the \textit{conpaas/scripts/create\_vm/ec2-setup-new-vm-image.sh} script to the instance:
  \begin{verbatim}
    chmod 0400 yourpublickey.pem
    scp -i yourpublickey.pem \
      conpaas/scripts/create_vm/ec2-setup-new-vm-image.sh \
      root@instancename.com:
  \end{verbatim}

\item Now, ssh to your instance:
  \begin{verbatim}
    ssh -i yourpublickey.pem root@your.instancename.com
  \end{verbatim}
  Run the \verb+ec2-setup-new-vm-image.sh+ script inside the instance.
  This script will install all of the dependencies of the manager and
  agent processes as well as create the necessary directory structure.

\item Clean the filesystem by removing the
  \verb+ec2-setup-new-vm-image.sh+ file and any other temporary files you might
  have created.

\item Go to the EC2 administration page at the AWS website, right
  click on the running instance and select ``\emph{Create Image (EBS
    AMI)}''.  This step will take several minutes. More information about this step
 can be found at
  \url{http://docs.amazonwebservices.com/AWSEC2/latest/UserGuide/index.html?Tutorial\_CreateImage.html}.

\item After the image has been fully created, you can return to the
  EC2 dashboard, right-click on your instance, and terminate it.
\end{enumerate}

\subsection{Create a Security Group}
\label{sec:secgroup}

An AWS security group is an abstraction of a set of firewall rules to
limit inbound traffic. The default policy of a new group is to deny
all inbound traffic. Therefore, one needs to specify a whitelist of
protocols and destination ports that are accessible from the outside. 
The following ports should be open for all running instances: 
\begin{itemize}
\item TCP ports 80, 443, 5555, 8000, 8080 and 9000 -- used by the Web Hosting service
\item TCP port 3306 -- used by the MySQL service
\item TCP ports 8020, 8021, 8088, 50010, 50020, 50030, 50060, 50070, 50075, 50090, 50105, 54310 and 54311 -- used by the Map Reduce service
\item TCP ports 4369, 14194 and 14195 -- used by the Scalarix service
\item TCP ports 8475, 8999 -- used by the TaskFarm service
\item TCP ports 32636, 32638 and 32640 -- used by the XtreemFS service
\end{itemize}

AWS documentation is available at
\url{http://docs.amazonwebservices.com/AWSEC2/latest/UserGuide/index.html?using-network-security.html}.

\section{Creating a ConPaaS image for OpenNebula}
\label{sec:oneimage}

The Web Hosting Service is capable of running over an OpenNebula
installation. This section describes the process of configuring
OpenNebula to run ConPaaS. You can skip this section if you plan to
deploy ConPaaS over Amazon Web Services.

To create an image for OpenNebula you can execute the script\\
\textit{conpaas/scripts/create\_vm/opennebula-create-new-vm-image.sh}
in any 64-bit Debian or Ubuntu machine. Please note that you will need to have
root privileges on such a system. In case you do not have root access to a
Debian or Ubuntu machine please consider installing a virtual machine using
your favorite virtualization technology, or running a Debian/Ubuntu instance in
the cloud.

\begin{enumerate}
\item Make sure your system has the following executables installed
  (they are usually located in \verb+/sbin+ or \verb+/usr/sbin+, so
  make sure these directories are in your \verb+$PATH+): % $
  \emph{dd parted losetup kpartx mkfs.ext3 tune2fs mount debootstrap
    chroot umount grub-install}
\item It is particularly important that you use Grub version 2. To
  install it:
  \begin{verbatim}
  sudo apt-get install grub2
  \end{verbatim}
\item Edit the
  \textit{conpaas/scripts/create\_vm/opennebula-create-new-vm-image.sh} script
  if necessary: there are two sections in the script that you might need
  to customize with parameters that are specific to your system. These
  sections are marked by comment lines containing the text "TO CUSTOMIZE:".
  There are comments explaining each customizable parameter. 
\item Obtain the id of the OpenNebula datastore you want to use by running
  \verb+onedatastore list+. In the following example, we will use "100" as our datastore id.
\item Execute the image generation script as root.
\item The script generates an image file called \verb+conpaas.img+ by default.
You can now register it in OpenNebula, replacing '100' in this example with the
datastore id obtained with \verb+onedatastore list+.

\vspace{10 mm}

\begin{verbatim}
  cat <<EOF > /tmp/conpaas-one.image
  NAME          = "Conpaas"
  PATH          = ${PWD}/conpaas.img
  PUBLIC        = YES
  DESCRIPTION   = "Conpaas vm image"
  EOF
  oneimage create /tmp/conpaas-one.image -d 100
\end{verbatim}
\end{enumerate}
% $>

\paragraph{If things go wrong}~\\

Note that if anything fails during the image file creation, the script will
stop and it will try to revert any change it has done. However, it might not
always reset your system to its original state. To undo everything the script
has done, follow these instructions:

\begin{enumerate}
\item The image has been mounted as a separate file system. Find the
  mounted directory using command \verb+df -h+. The directory should
  be in the form of \verb+/tmp/tmp.X+.
  
\item There may be a \verb+dev+ and a \verb+proc+ directories mounted
  inside it. Unmount everything using:
  \begin{verbatim}
    sudo umount /tmp/tmp.X/dev /tmp/tmp.X/proc /tmp/tmp.X
  \end{verbatim}
  
\item Find which loop device your using:
  \begin{verbatim}
    sudo losetup -a
  \end{verbatim}
  
\item Remove the device mapping:
  \begin{verbatim}
    sudo kpartx -d /dev/loopX
  \end{verbatim}
  
\item Remove the binding of the loop device:
  \begin{verbatim}
    sudo losetup -d /dev/loopX
  \end{verbatim}

\item Delete the image file 

\item Your system should be back to its original state.
\end{enumerate}

\subsection{Make sure OpenNebula is properly configured}

OpenNebula's OCCI daemon is used by ConPaaS to communicate with your
OpenNebula cluster.

\begin{enumerate}
\item Ensure your occi-server.conf contains the following lines in instance\_types:
\begin{verbatim}
:custom:
  :template: custom.erb
\end{verbatim}
\item At the end of the OCCI profile file \verb+occi_templates/common.erb+ 
  from your OpenNebula installation, add the content of the file
  \verb+misc/common.erb+ from the ConPaaS distribution. This new version 
  features a number of improvements from the standard version:
  \begin{itemize}
  \item The match for \verb+OS TYPE:arch+ allows the caller to specify
    the architecture of the machine.
  \item The graphics line allows for using vnc to connect to the VM.
    This is very useful for debugging purposes and is not necessary
    once testing is complete.

  \end{itemize}

\item Make sure you started OpenNebula's OCCI daemon

\end{enumerate}

\section{Miscellaneous}
\label{sec:misc}

\subsection{The credit system}

The frontend is designed to maintain accounting of resources used by
each user. When a new user is created, she receives a number of
credits as specified in the ``main.ini'' configuration file. Later on,
one credit is subtracted each time a VM is executed for (a fraction
of) one hour. The administrator can change the number of credits by
directly editing the frontend's database. 

%\subsection{Application sandboxing}
%
%The default ConPaaS configuration creates strong sandboxing so that
%applications cannot open sockets, access the file system, execute
%commands, etc. This makes the platform relatively secure against
%malicious applications. On the other hand, it strongly restricts the
%actions that ConPaaS applications can do. To reduce these security
%measures to a more usable level, you need to edit two files:
%
%\begin{itemize}
%\item To change restrictions applied to PHP applications, edit file
%  \verb+web-servers/etc/fpm.tmpl+ to change the list of
%  \verb+disable_functions+. Do not forget to recreate a file
%  \verb+ConPaaSWeb.tar.gz+ out of the entire \verb+web-servers+
%  directory, and to copy it at the URL specified in file
%  \verb+frontend/conf/manager-user-data+.
%\item To change restrictions applied to Java applications, edit file
%  ``web-servers/etc/tomcat-catalina.policy''. Do not forget to
%  recreate a file ConPaaSWeb.tar.gz out of the entire ``web-servers''
%  directory, and to copy it at the URL specified in file
%  ``frontend/conf/manager-user-data''.
%\end{itemize}

\section{About this document}
\label{sec:about}


\begin{verbatim}
Copyright (c) 2010-2013, Contrail consortium.
All rights reserved.

Redistribution and use in source and binary forms, 
with or without modification, are permitted provided
that the following conditions are met:

 1. Redistributions of source code must retain the
    above copyright notice, this list of conditions
    and the following disclaimer.
 2. Redistributions in binary form must reproduce
    the above copyright notice, this list of 
    conditions and the following disclaimer in the
    documentation and/or other materials provided
    with the distribution.
 3. Neither the name of the Contrail consortium nor the
    names of its contributors may be used to endorse
    or promote products derived from this software 
    without specific prior written permission.

THIS SOFTWARE IS PROVIDED BY THE COPYRIGHT HOLDERS AND
CONTRIBUTORS "AS IS" AND ANY EXPRESS OR IMPLIED WARRANTIES,
INCLUDING, BUT NOT LIMITED TO, THE IMPLIED WARRANTIES OF
MERCHANTABILITY AND FITNESS FOR A PARTICULAR PURPOSE ARE
DISCLAIMED. IN NO EVENT SHALL THE COPYRIGHT HOLDER OR
CONTRIBUTORS BE LIABLE FOR ANY DIRECT, INDIRECT, INCIDENTAL,
SPECIAL, EXEMPLARY, OR CONSEQUENTIAL DAMAGES (INCLUDING, 
BUT NOT LIMITED TO, PROCUREMENT OF SUBSTITUTE GOODS OR 
SERVICES; LOSS OF USE, DATA, OR PROFITS; OR BUSINESS 
INTERRUPTION) HOWEVER CAUSED AND ON ANY THEORY OF LIABILITY,
WHETHER IN CONTRACT, STRICT LIABILITY, OR TORT
(INCLUDING NEGLIGENCE OR OTHERWISE) ARISING IN ANY WAY OUT
OF THE USE OF THIS SOFTWARE, EVEN IF ADVISED OF THE
POSSIBILITY OF SUCH DAMAGE.
\end{verbatim}


\end{document}


