\documentclass[10pt]{article}

\usepackage{hyperlatex} % for HTML generation

\begin{ifhtml}
  
  \begin{ifclear}{noframe}
    \usepackage[simple]{frames}     % for building HTML frames
  \end{ifclear}

  \htmlcss{conpaas.css}
  \providecommand{\url}[1]{\xlink{#1}{#1}}
  \setcounter{htmlautomenu}{4}
  \setcounter{secnumdepth}{2} 
  \setcounter{htmldepth}{1}
  
  \htmladdress{Copyright \copyright 2011-2012 \xlink{Contrail}{http://www.contrail-project.eu} consortium.\\
    All rights reserved.}

  \renewcommand{\maketitle}{\EmptyP{\HlxTitleP}{
      \HlxBlk\xml{div style="width: 78%;"}\xml{frontmatter}
      \xml{h1 align=center}\HlxTitle\xml{/h2}
      \EmptyP{\HlxAuthorP}{\xml{h2 align=center}\HlxAuthor\xml{/h2}}{}
      \EmptyP{\HlxDate}{\xml{h2 align=center}\HlxDate\xml{/h2}}{}
      \xml{/frontmatter}
    }{}}

  \renewcommand{\HlxFramesDescription}[2]{
    \xml{frameset rows="100%" cols="25%,75%"}
      \xml{frame src="#1_toc#2" name="contents" marginwidth="5"
        marginheight="5" border="0"}
      \xml{frame src="#1_0#2" name="main" marginwidth="20" 
        marginheight="20"}
      \xml{noframes}
      This document uses frames to assist navigation.
      Your browser is currently not supporting the use of frames, but you 
      may still access the 
      \xml{a target="_top" href="#1_0#2"}non-framed version\xml{/a}.
      \xml{/noframes}
      \xml{/frameset}}
    
    \renewcommand{\HlxFramesNavigation}{%
      \HlxTocName
      \htmlpanel{0}
      \HlxSection{-5}{}*{\navigationname}
      \xml{base target="main"}
      \xml{div style="width: 26%;"}
      \htmlmenu[0]{2}
      \renewcommand{\bottommatter}{}}

\end{ifhtml}


\T\usepackage{hyperref}
\T\usepackage{url}


\title{ConPaaS -- User Guide\htmlonly{ [\xml{a href="userguide.pdf"}pdf\xml{/a}]}}
\htmltitle{ConPaaS -- User Guide}
\author{Ismail El Helw \and Guillaume Pierre}
\date{ConPaaS-0.9.0}


\begin{document}

\maketitle

\T\vfil
\T\tableofcontents
\T\vfil
\T\newpage

\section{Introduction}

\xlink{ConPaaS}{http://www.conpaas.eu} is an open-source runtime
environment for hosting applications in the cloud which aims at
offering the full power of the cloud to application developers while
shielding them from the associated complexity of the cloud.

ConPaaS is designed to host both high-performance scientific
applications and online Web applications. It runs on a variety of
public and private clouds, and is easily extensible.  ConPaaS
automates the entire life-cycle of an application, including
collaborative development, deployment, performance monitoring, and
automatic scaling. This allows developers to focus their attention on
application-specific concerns rather than on cloud-specific details.

ConPaaS is organized as a collection of \textbf{services}, where each
service acts as a replacement for a commonly used runtime environment.
For example, to replace a MySQL database, ConPaaS provides a
cloud-based MySQL service which acts as a high-level database
abstraction. The service uses real MySQL databases internally, and
therefore makes it easy to port a cloud application to ConPaaS.
Unlike a regular centralized database, however, it is self-managed and
fully elastic: one can dynamically increase or decrease its processing
capacity by requesting it to reconfigure itself with a different
number of virtual machines.

ConPaaS currently contains six services: 

\begin{itemize}
\item \textbf{Two Web hosting services} respectively specialized for
  hosting PHP and JSP applications;
\item \textbf{MySQL database} service;
\item \textbf{Scalarix service} offering a scalable in-memory
  key-value store;
\item \textbf{MapReduce service} providing thewell-known
  high-performance computation framework;
\item \textbf{TaskFarming service} high-performance batch processing.
\end{itemize}

ConPaaS applications can be composed of any number of services. For
example, a bio-informatics application may make use of a PHP and a
MySQL service to host a Web-based frontend, and link this frontend to
a MapReduce backend service for conducting high-performance genomic
computations on demand.

\section{Usage overview}

Most operations in ConPaaS can be done using the ConPaaS frontend,
which gives a Web-based interface to the system. The front-end allows
users to register, create services, upload code and data to the
services, and configure each service. 

\begin{itemize}
\item The Dashboard page displays the list of services currently
  active in the system. 
\item Each service comes with a separate page which allows one to
  configure it, upload code and data, and scale it up and down.
\end{itemize}

All the functionalities of the frontend are also available using a
command-line interface. This allows one to script commands for
ConPaaS. The command-line interface also features additional advanced
functionalities, which are not available using the front-end.

\subsection{Controlling services using the front-end}

The ConPaaS front-end provides a simple and intuitive interface for
controlling services. We discuss here the features that are common to
all services, and refer to the next sections for service-specific
functionality.

\begin{description}
\item[Create a service.] Click on ``create new service'', then select
  the service you want to create. This operation starts a new
  ``Manager'' virtual machine instance. The manager is in charge of
  taking care of the service, but it does not host applications
  itself. Other instances in charge of running the actual application
  are called ``agent'' instances.
\item[Start a service.] Click on ``start'', this will create a new
  virtual machine which can host applications, depending on the type
  of service.
\item[Rename the service.] By default all new services are named ``New
  service.'' To give a meaningful name to a service, click on this
  name in the service-specific page and enter a new name.
\item[Check the list of virtual instances.] A service can run using
  one or more virtual machine instances. The service-specific page
  shows the list of instances, their respective IP addresses, and the
  role each instance is currently having in the service. Certain
  services use a single role for all instances, while other services
  specialize different instances to take different roles. For example,
  the PHP Web hosting service distinguishes three roles: load
  balancers, web servers, and PHP servers.
\item[Scale the service up and down.] When a service is started it
  uses a single ``agent'' instance. To add more capacity, or to later
  reduce capacity you can vary the number of instances used by the
  service. Click the numbers below the list of instances to request
  adding or removing servers. The system reconfigures itself without
  any service interruption.
\item[Stop the service.] When you do not need to run the application
  any more, click ``stop'' to stop the service. This stops all
  instances except the manager which keeps on running. 
\item[Terminate the service.] Click ``terminate'' to terminate the
  service. At this point all the state of the service manager will be
  lost.
\end{description}

\subsection{Controlling services using the command-line interfaces}

Command-line interfaces allow one to control services without using
the graphical interface. The command-line interfaces also offer
additional functionality for advanced usage of the services.

Each service can be controlled using its own command-line tool:

\begin{itemize}
\item The two Web hosting services are controlled using \texttt{cpsclient.web}
\item The MySQL service is controlled using \texttt{cpsclient.mysql}
\item The Scalarix service is controlled using \texttt{cpsclient.scalarix}
\item The MapReduce service is controlled using cpsclient.mapreduce
\end{itemize}

To use these tools you must first set the PYTHONPATH variable to
contain the full path name of the ``src'' subdirectory within the
ConPaaS directory:
\begin{verbatim}
$ export PYTHONPATH=/path/to/conpaas/src
\end{verbatim}
% $

The \texttt{cpsclient.*} tools always take the URL of the service
manager as its first argument. This URL is provided by the front-end.

\vspace{1em}

\begin{description}
\item[List all options of the command-line tool.]~
\begin{verbatim}
$ ./cpsclient.servicename help
\end{verbatim}
%$

\item[Start the service.]~
\begin{verbatim}
$ ./cpsclient.servicename http://x-x-x-x/ startup
\end{verbatim}
%$

\item[Stop the service.]~
\begin{verbatim}
$ ./cpsclient.servicename http://x-x-x-x/ shutdown
\end{verbatim}
%$

\item[Scale the service up and down.]~
\begin{verbatim}
$ ./cpsclient.servicename http://x-x-x-x/ add_nodes -h
$ ./cpsclient.servicename http://x-x-x-x/ remove_nodes -h
$ ./cpsclient.web http://x-x-x-x/ add_nodes -w 1 -b 1
$ ./cpsclient.web http://x-x-x-x/ remove_nodes -w 1 -b 1
\end{verbatim}

\end{description}

\subsection{The credit system}

In Cloud computing, resources come at a cost. ConPaaS reflects this
reality in the form of a credit system. Each user is given a number of
credits that she can use as she wishes. One credit corresponds to one
hour of execution of one virtual machine. The number of available
credits is always mentioned in the top-right corner of the front-end.
Once credits are exhausted, your running instances will be stopped and
you will not be able to use the system until the administrator decides
to give additional credit.

Note that every service consumes credit, even if it is in ``stopped''
state. The reason is that stopped services still have one ``manager''
instance running. To stop using credits you must completely terminate
your services.

\section{Tutorial: hosting WordPress in ConPaaS}

This short tutorial illustrates the way to use ConPaaS to install and
host WordPress (\url{http://www.wordpress.org}), a well-known third-party Web
application. WordPress is implemented in PHP using a MySQL database so
we will need a PHP and a MySQL service in ConPaaS.

\begin{enumerate}
\item Open the ConPaaS front-end in your Web browser and log in. If
  necessary, create yourself a user account and make sure that you
  have at least 5 credits. Your credits are always shown in the
  top-right corner of the front-end. One credit corresponds to one
  hour of execution of one virtual machine instance.
\item Create a MySQL service, start it, reset its password. Copy the
  IP address of the master node somewhere, we will need it in step 5.
\item Create a PHP service, start it.
\item Download a Wordpress tarball from \url{http://www.wordpress.org}, and
  expand it in your computer.
\item Copy file \texttt{wordpress/wp-config-sample.php} to
  \texttt{wordpress/wp-config.php} and edit the \texttt{DB\_NAME},
  \texttt{DB\_USER}, \texttt{DB\_PASSWORD} and \texttt{DB\_HOST}
  variables to point to the database service. You can choose any
  database name for the \texttt{DB\_NAME} variable as long as it does
  not contain any special character. We will reuse the same name in
  step 7.
\item Rebuild a tarball of the directory such that it will expand in
  the current directory rather than in a \texttt{wordpress}
  subdirectory. Upload this tarball to the PHP service, and make the
  new version active. 
\item Connect to the database using the command proposed by the
  frontend. Create a database of the same name as in step 5 using
  command "\texttt{CREATE DATABASE databasename;}"
\item Open the page of the PHP service, and click ``access
  application.'' Your browser will display nothing because the
  application is not fully installed yet. Visit the same site at URL
  \texttt{http://xxx.yyy.zzz.ttt/wp-admin/install.php} and fill in the
  requested information (site name etc).
\item That's it! The system works, and can be scaled up and down. 
\end{enumerate}

Note that the ``file upload'' functionality of WordPress will not work
if you scale the system up. This is because WordPress stores files in
the local file system of the PHP server where the upload has been
processed. If a subsequent request for this file is processed by
another PHP server then the file will not be found. In a next ConPaaS
release we wil provide a shared file system service which will allow
one to avoid this issue.


\section{The PHP Web hosting service}

The PHP Web hosting service is dedicated to hosting Web applications
written in PHP. It can also host static Web content.

\subsection{Uploading application code}

PHP applications can be uploaded in the form of a \texttt{tar} or
\texttt{zip} archive.  Attention: the archive must expand \emph{in the
  current directory} rather than in a subdirectory. The service does
not immediately use new applications when they are uploaded. The
frontend shows the list of versions that have been uploaded; choose
one version and click ``make active'' to activate it.

Note that the frontend only allows uploading archives smaller than a
certain size.  To upload large archives, you must use the command-line
tool:
\begin{verbatim}
$ ./cpsclient.web http://x-x-x-x/ upload_code_version path/to/archive.zip
\end{verbatim}
%$

\subsection{Access the application}

The frontend gives a link to the running application. This URL will
remain valid as long as you do not stop the service.

\subsection{Using PHP sessions}

PHP normally stores session state in its main memory. When scaling up
the PHP service, this creates problems because multiple PHP servers
running in different VM instances cannot share their memory. To
support PHP sessions the PHP service features a key-value store where
session states can be transparently stored. To overwrite PHP session
functions such that they make use of the shared key-value store, the
PHP service includes a standard ``phpsession.php'' file at the
beginning of every .php file of your application that uses sessions,
i.e. in which function session\_start() is encountered.  This file
overwrites the session handlers using the
session\_set\_save\_handler() function.

This modification is transparent to your application so no particular
action is necessary to use PHP sessions in ConPaaS.

\subsection{Debug mode}

By default the PHP service does not display anything in case PHP
errors occur while executing the application. This setting is useful
for production, when you do not want to reveal internal information to
external users. While developing an application it is however useful
to let PHP display erors.
\begin{verbatim}
$ ./cpsclient.web http://x-x-x-x/ update_php_configuration \
  -p "display_errors=On" 
\end{verbatim}
%$

\section{The Java Web hosting service}

The Java Web hosting service is dedicated to hosting Web applications
written in Java using JSP or servlets. It can also host static Web
content.

\subsection{Uploading application code}

Applications in the Java Web hosting service can be uploaded in the
form of a \texttt{war} file. The service does not immediately use new
applications when they are uploaded. The frontend shows the list of
versions that have been uploaded; choose one version and click ``make
active'' to activate it.  

Note that the frontend only allows uploading archives smaller than a
certain size.  To upload large archives, you must use the command-line
tool.
\begin{verbatim}
$ ./cpsclient.web http://x-x-x-x/ upload_code_version path/to/archive.war
\end{verbatim}
%$

\subsection{Access the application}

The frontend gives a link to the running application. This URL will
remain valid as long as you do not stop the service.


\section{The MySQL database service}

The MySQL service provides the famous database in the form of a
ConPaaS service. When scaling the service up and down, it creates (or
deletes) database replicas using the master-slave mechanism. At the
moment, the service does not implement load balancing of database
queries between the master and its slaves. Replication therefore
provides fault-tolerance properties but no performance improvement.

\subsection{Resetting the user password}

When a MySQL service is started, a new user \texttt{mysqldb} is
created with a randomly-generated password. To gain access to the
database you must first reset this password. Click ``Reset password''
in the front-end, and choose the new password.

Note that the user password is \emph{not} kept by the ConPaaS
frontend. If you forget the password the only thing you can do is
reset the password again to a new value.

\subsection{Accessing the database} 

The frontend provides the command-line to access the database.
Copy-paste this command in a terminal. You will be asked for the user
password, after which you can use the database as you wish.

Note that the \texttt{mysqldb} user has extended priviledges. It can
create new databases, new users etc.

\subsection{Uploading a database dump}

The ConPaaS frontend allows to easily upload database dumps to a MySQL service. Note that this functionality is restricted to dumps of a relatively small size. To upload larger dumps you can always use the regular mysql command for this:
\begin{verbatim}
$ mysql mysql-ip-address -u mysqldb -p < dumpfile.sql
\end{verbatim}
% $

\section{The Scalarix key-value store service}

The Scalarix service provices an in-memory key-value store. It is
highly scalable and fault-tolerant. This service deviates slightly
from the organization of other services in that it does not have a
separate manager virtual machine instance. Scalarix is fully symmetric
so any scalarix node can act as a service manager.

\subsection{Accessing the key-value store}

Clients of the Scalarix service need the IP address of (at least) one
node to connect to the servicve. Copy-paste the address of any of the
running instances in the client. A good choice is the first instance
in the list: when scaling the service up and down, other instances may
be created or removed. The first instance will however remain across
these reconfigurations, until the service is terminated.



\subsection{Managine the key-value store}

Scalarix provides its own Web-based interface to monitor the state and
performance of the key-value store, manually add or query key-value
pairs, etc. For convenience reasons the ConPaaS front-end provides a
link to this interface.

\section{The MapReduce service}

The MapReduce service provides the well-known Apache Hadoop framework
in ConPaaS. Once the MapReduce service is created and started, the
front-end provides useful links to the Hadoop namenode, the job
tracker, and to a graphical interface which allows to upload/download
data to/from the service and issue MapReduce jobs.

\section{The TaskFarm service}

The TaskFarm service provides a bag of tasks scheduler for ConPaaS.
The user needs to provide a list of independent tasks to be executed
on the cloud and a file system location where the tasks can read 
input data and/or write output data to it. The service first enters 
a sampling phase, where its agents sample the runtime of the given tasks on 
different cloud instances. The service then based on the sampled runtimes, 
provides the user with a list of schedules.
Schedules are presented in a graph and the user can choose between 
cost/makespan of different schedules for the given set of tasks.
After the choice is made the service enters the execution phase and 
completes the execution of the rest of the tasks according to the user's
choice.

\subsection{The bag of tasks file}
The bag of tasks file is a simple plain text file that contains the 
list of tasks along with their arguments to be executed. The tasks 
are seperated by new lines. This file needs to be uploaded to the 
service, before the service can start sampling. Below is an example of a simple bag of tasks file containing three tasks:
\texonly{\rule{\linewidth}{0.5mm}}
\htmlrule
\begin{verbatim}
/bin/sleep 1 && echo "slept for 1 seconds" >> /mnt/xtreemfs/log
/bin/sleep 2 && echo "slept for 2 seconds" >> /mnt/xtreemfs/log
/bin/sleep 3 && echo "slept for 3 seconds" >> /mnt/xtreemfs/log
\end{verbatim}
\texonly{\rule{\linewidth}{0.5mm}}
\htmlrule

\subsection{The filesystem location}

TaskFarm service uses XtreemFS for data input/output. The actual task 
code can also reside in the XtreemFS. The user can optionally provide
an XtreemFS location which is then mounted on TaskFarm agents.

\subsection{The demo mode}

With large bags of tasks and/or with long running tasks, the TaskFarm 
service can take a long time to execute the given bag.
The service provides its users with a progress bar and reports the amount 
money spent so far. TaskFarm service also provides a ``demo'' mode where 
the users can try the service with custom bags without spending time and 
money.

\section{Building new types of services}

The architecture of ConPaaS allows developers to build new types of
services. To learn how to do this, please check the
``\xlink{Internals}{internals.pdf}'' ConPaaS documentation.

\section{About this document}

\begin{verbatim}
Copyright (c) 2010-2012, Contrail consortium.
All rights reserved.

Redistribution and use in source and binary forms, 
with or without modification, are permitted provided
that the following conditions are met:

 1. Redistributions of source code must retain the
    above copyright notice, this list of conditions
    and the following disclaimer.
 2. Redistributions in binary form must reproduce
    the above copyright notice, this list of 
    conditions and the following disclaimer in the
    documentation and/or other materials provided
    with the distribution.
 3. Neither the name of the Contrail consortium nor the
    names of its contributors may be used to endorse
    or promote products derived from this software 
    without specific prior written permission.

THIS SOFTWARE IS PROVIDED BY THE COPYRIGHT HOLDERS AND
CONTRIBUTORS "AS IS" AND ANY EXPRESS OR IMPLIED WARRANTIES,
INCLUDING, BUT NOT LIMITED TO, THE IMPLIED WARRANTIES OF
MERCHANTABILITY AND FITNESS FOR A PARTICULAR PURPOSE ARE
DISCLAIMED. IN NO EVENT SHALL THE COPYRIGHT HOLDER OR
CONTRIBUTORS BE LIABLE FOR ANY DIRECT, INDIRECT, INCIDENTAL,
SPECIAL, EXEMPLARY, OR CONSEQUENTIAL DAMAGES (INCLUDING, 
BUT NOT LIMITED TO, PROCUREMENT OF SUBSTITUTE GOODS OR 
SERVICES; LOSS OF USE, DATA, OR PROFITS; OR BUSINESS 
INTERRUPTION) HOWEVER CAUSED AND ON ANY THEORY OF LIABILITY,
WHETHER IN CONTRACT, STRICT LIABILITY, OR TORT
(INCLUDING NEGLIGENCE OR OTHERWISE) ARISING IN ANY WAY OUT
OF THE USE OF THIS SOFTWARE, EVEN IF ADVISED OF THE
POSSIBILITY OF SUCH DAMAGE.
\end{verbatim}

\end{document}

