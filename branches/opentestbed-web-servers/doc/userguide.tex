\documentclass[12pt]{article}
\usepackage{listings}
\usepackage{framed}


\newenvironment{what}
{\begin{description} \item [What is happening now?] \hfill \\}
{\end{description}}

\newenvironment{framedbox}[1]%
{\begin{framed}
 \begingroup
 \fontsize{#1}{#1}\selectfont
}
{
 \endgroup
 \end{framed}
}

\pagestyle{myheadings}
\markright{User Manual:ConPaaS Web Hosting Service}

\begin{document}
\title{User Manual: ConPaaS Web Hosting Service}
\date{August 15, 2011}
\thispagestyle{empty}

\begin{center}
\begingroup
\fontsize{20pt}{20pt}\selectfont
\textbf{User Manual: ConPaaS Web Hosting Service} \linebreak
\endgroup

\begingroup
\fontsize{16pt}{16pt}\selectfont
August 15, 2011
\endgroup
\end{center}

\section{Cloud Front-end}
The cloud front-end provides an intuitive web-based user interface that
allows users to register new accounts in order to start using the Web
Hosting Service. Registered users can create as well as terminate
services. Additionally, the front-end provides a simplified interface
through which a user can configure his services.

\subsection{ConPaaS Open Testbed}
The ConPaaS Open Testbed is a platform where users can experiment with ConPaaS
services for free. The testbed utilizes compute resources from Amazon Web
Services and places additional usage restrictions that are not part of the
standard ConPaaS release. These restrictions guarantee that malicious
applications do not hinder the testbed or cause harm to other entities over the
Internet.
\begin{itemize}
  \item User applications cannot open files, use sockets or start processes.
  \item Users will be limited to a certain number of hours of usage.
\end{itemize}
Registered users are assigned a certain amount of credits by the system
administrator. Each service runs on behalf of a user consumes a specific
amount of credits depending on its size. One credit is worth running a single
Amazon EC2 instance for a full hour. Note that a service may be utilizing
multiple EC2 instances concurrently and hence charge multiple credits per hour.
A full hour will be accounted for partial hour usage per EC2 instance. If a
user runs out of credits, all his services will be terminated. The amount of
credit assigned to a user is displayed at the top right of the web pages of
the front-end.

\subsection{Register a user}
Note: User registration can only be done through the cloud front-end.
\begin{enumerate}
\item Click on the "register" link.
\item Enter your user name and click on the "register" button.
\item You will be automatically logged in. Notice your user name and the "logout"
      link at the top right corner of the web page. You are now ready to
      experiment with the Web Hosting Service.
\end{enumerate}

\subsection{Create a Web Hosting Service}
Node: Creating a new service can only be done through the cloud front-end. 
\begin{enumerate}
\item Starting from the user home page, click on the "create service" button.
\item Select a type of web hosting service. For example "PHP Service".
\item Select a target cloud platform. For example "Amazon EC2".
\item Click on the "create service" button to start.
 \begin{what}
  The Web interface requested a new virtual machine from Amazon EC2 and
  is waiting for it to boot. This virtual machine will be the managing
  node of the new service. The web page will be automatically redirected
  and a list of all of the created services will be displayed.
  Note that, the service status is "Initializing". When the service is
  ready, its status will change to "created" and you can click on it to
  configure it further.
 \end{what}
\end{enumerate}

\subsection{Rename the Service}
Renaming a service can only be done through the cloud front-end. 
\begin{enumerate}
\item Starting from the user home page, click on the service you intend to
      rename.
\item Click on the service name at the top left of the web
      page. A dialog box will appear where you can enter the new service
      name.
\end{enumerate}

\subsection{Terminate the service}
Note: terminating the service can only be done through the cloud front-end.
\begin{enumerate}
\item Starting from the user home page, click on the service you intend to
      terminate.
\item Press on the "terminate" button at the top right of the page.
      Terminating the service will release the virtual machine hosting
      the manager and will delete all of the service's configuration and
      uploaded code.
\end{enumerate}

\section{Starting and Stopping the Service}

\subsection{Start the Service}
\subsubsection{Through the Front-end}
\begin{enumerate}
\item Starting from the user home page, click on the service you intend to start.
\item Notice the message "No instances are running". This means that there is no
      web server running yet.
\item Click on the "start" button at the top right
      of the page to start a web server. Notice the progress message that
      appeared at the top of the page.
      \begin{what}
        The service requested a new virtual machine from the cloud provider.
        When the machine is ready, the manager will configure it to run
        a web server. When the web server is ready, the web page will
        display the running instances' information.
      \end{what}
\item Notice the displayed instance information. It is tagged with "proxy",
      "web" and "php" which means that this virtual machine is running a proxy
      (load balancer) server, a web server and it supports executing PHP
      scripts.On the right end you will find the domain name of the virtual
      machine. You can use this to access the virtual machine directly.
\item Notice the link labeled "access active version". Click on it to access
      the newly created web server. The web servers start with a default
      welcome page.
\end{enumerate}

\subsubsection{Using the Command-line Client}
\begin{framedbox}{8pt}\begin{verbatim}
$ ./cpsclient.web http://x-x-x-x/ startup
\end{verbatim}\end{framedbox}


\subsection{Stop the Service}
\subsubsection{Through the Front-end}
\begin{enumerate}
\item Starting from the user home page, click on the service you intend to stop.
\item Press on the "stop" button at the top right of the web page.
\item Stopping the service would release the web servers but the service
      manager will remain active. If you want to permanently destroy the
      service, press on the terminate button after you stop the service.
\end{enumerate}
\subsubsection{Using the Command-line Client}
\begin{framedbox}{8pt}\begin{verbatim}
$ ./cpsclient.web http://x-x-x-x/ shutdown
\end{verbatim}\end{framedbox}

\section{Code Management}
The Web Hosting Service can manage and store multiple code archives. You can
upload code archives to it and select which one should be active online.
This section explains how to manage code archives.

\subsection{Upload Code Version}
\subsubsection{Through the Front-end}
\begin{enumerate}
\item Use the "choose file" button to upload a code archive. When creating an
      archive, you need to make sure it expands directly in the same directory.
      The upload file must be an archive of type '.zip', '.tar', '.tar.bz2' or
      '.tar.gz'. PHP applications should have a file named "index.php"
      which will be the default page. Java applications should have a file
      named index.jsp.
\item Notice that the "available code versions" list grew. A new code version
      appeared in the list but it is not active yet. Hover over the new code
      version with the mouse and two more links will appear; "set active" and
      "download".
\end{enumerate}


\subsubsection{Using the Command-line Client}
\begin{framedbox}{8pt}\begin{verbatim}
$ ./cpsclient.web http://x-x-x-x/ upload_code_version -h
Usage: upload_code_version <filename>

Options:
  -h, --help  show this help message and exit

$ ./cpsclient.web http://x-x-x-x/ upload_code_version path/to/archive.zip
codeVersionId: code-XXXXX
\end{verbatim}\end{framedbox}

\subsection{Activate Code Version}
\subsubsection{Through the Front-end}
\begin{enumerate}
\item Hover over a code version with the mouse and two links will appear;
      "set active" and "download".
\item Click on "set active" to activate this version online.
\item Notice that the selected code version is labeled with "active".
\item If the service is already running, click on "access active version" to
      validate that the new code version is running. Your web browser would
      normally cache web pages so you may need to refresh the page to view
      the latest updates.
\end{enumerate}

\subsubsection{Using the Command-line Client}
\begin{framedbox}{8pt}\begin{verbatim}
$ ./cpsclient.web http://x-x-x-x/ update_java_configuration -h
Usage: update_java_configuration

Options:
  -h, --help            show this help message and exit
  -c CODEVERSIONID, --code=CODEVERSIONID

$ ./cpsclient.web http://x-x-x-x/ update_java_configuration -c code-XXXX
\end{verbatim}\end{framedbox}

\subsection{Download Code Version}
\subsubsection{Through the Front-end}
\begin{enumerate}
\item Hover over a code version with the mouse and two links will appear;
      "set active" and "download".
\item Click the "download" link will download the file to your local computer.
\end{enumerate}

\section{Resource Management}
One of the advantages of ConPaaS is elasticity. The Web Hosting Service can
configure multiple servers and assign them different roles to scale. The
deployment can grow and shrink transparently to the users without any
service disruption.

\subsection{Scaling Out/In}
\subsubsection{Through the Front-end}
\begin{enumerate}
\item Notice the section labeled "add or remove instances to your deployment" in
      the web page were there are 3 boxes labeled "proxy", "web" and "php" with
      a 0 to the left of each one.
\item Click on the 0 of any box and a dialog will appear where you can specify
the number of nodes you want to add/remove.
\item Let's add 1 web server, 1 proxy and 1 php. Then press on the "submit"
      button to their right.
\item A progress message will show up at the top of the page.
 \begin{what}
  Requesting new virtual machines will take some time. As soon as the new
  virtual machines become available, the manager will configure them and
  reconfigure the old nodes as well. If you want to monitor the progress of the
  new virtual machines more closely, click on the "raw log" link at the top of
  the page to view the log produced by the manager. You will need to refresh
  this page to view recent updates. Once the nodes are ready, they will be
  displayed on the web page.
 \end{what}
\item Notice that the web page is now displaying the newly created nodes as
      well. Each node is tagged with its roles (proxy, web or php).
\end{enumerate}

\subsubsection{Using the Command-line Client}
\begin{framedbox}{8pt}\begin{verbatim}
$ ./cpsclient.web http://x-x-x-x/ add_nodes -h
Usage: add_nodes

Options:
  -h, --help            show this help message and exit
  -p PROXY, --proxy=PROXY
  -w WEB, --web=WEB     
  -b BACKEND, --backend=BACKEND

$ ./cpsclient.web http://x-x-x-x/ add_nodes -w 1 -b 1

$ ./cpsclient.web http://x-x-x-x/ remove_nodes -h
Usage: remove_nodes

Options:
  -h, --help            show this help message and exit
  -p PROXY, --proxy=PROXY
  -w WEB, --web=WEB     
  -b BACKEND, --backend=BACKEND

$ ./cpsclient.web http://x-x-x-x/ remove_nodes -w 1 -b 1
\end{verbatim}\end{framedbox}

\section{Command-line Administration}
You can perform all of the operations provided by the web interface
by using a command-line tool. \textbf{\emph{Prerequisites: python $>=$ 2.6,
python-pycurl and python-simplejson packages.}}
Create a service, go to its web page and copy the URL provided by the 
"access manager" link at the top of the page. This URL points to the manager
directly and you can use it with the command-line program "cpsclient.web" to
issue commands.

\subsection{Prepare Command-line Environment}
\begin{itemize}
\item Download the source code file ConPaaSWeb.tar.gz.
\item Unpack it and prepare your environment as follows:
\end{itemize}
\begin{framedbox}{12pt}\begin{verbatim}
$ tar -zxf ConPaaSWeb.tar.gz             # unpack the archive
$ export PYTHONPATH=`pwd`/ConPaaSWeb/src # Set PYTHONPATH
\end{verbatim}\end{framedbox}
\begin{itemize}
\item The PYTHONPATH environment variable needs to be pointing to the
      location of the 'src' directory on your file system.
\item Run ConPaaSWeb/bin/cpsclient.web to view a list of supported
      operations.
\end{itemize}
\begin{framedbox}{8pt}\begin{verbatim}
$ ./cpsclient.web
Usage: ./cpsclient.web URL ACTION options

Action could be one of:
 ACTION                    DESCRIPTION
 add_nodes                   Add more service nodes to a deployment
 getLog                      Get raw logging
 get_configuration           Get the configuration of a deployment
 get_node_info               Get information about a single service node
 get_service_history         Get the state change history of a deployment
 get_service_info            Get the state of a deployment
 get_service_performance     Get the average request rate and throughput
 help                        Print the help menu
 list_code_versions          List identifiers of all code versions stored by a deployment
 list_nodes                  Get a list of service nodes
 remove_nodes                Remove some service nodes from a deployment
 shutdown                    Shutdown a deployment
 startup                     Startup a deployment
 update_java_configuration   Update the configuration of a Java deployment
 update_php_configuration    Update the configuration of a PHP deployment
 upload_code_version         Upload a new code version
\end{verbatim}\end{framedbox}

\begin{itemize}
\item Use the "access manager" URL as a first argument to cpsclient.web followed
      by one of the operation names to perform it. Use the '-h' option to check
      if an operation requires additional arguments.
\end{itemize}
\begin{framedbox}{8pt}\begin{verbatim}
$ ./cpsclient.web http://x-x-x-x/ get_node_info -h
Usage: get_node_info <nodeId>

Options:
  -h, --help  show this help message and exit

$ ./cpsclient.web http://x-x-x-x/ get_node_info i-23dffe4
Service Node         Address              Role(s)
i-23dffe4d           ec2-xxx-xx-xx-xxx.compute-x.amazonaws.com WEB
\end{verbatim}\end{framedbox}


\end{document}

