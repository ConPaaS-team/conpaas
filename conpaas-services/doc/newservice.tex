\documentclass[a4paper]{article}
\usepackage{framed}
\usepackage{dirtree}

\usepackage{listings}
\usepackage{color}
\usepackage{textcomp}
\definecolor{listinggray}{gray}{0.9}
\definecolor{lbcolor}{rgb}{0.9,0.9,0.9}
\lstset{
	backgroundcolor=\color{lbcolor},
	tabsize=4,
	rulecolor=,
	language=python,
        basicstyle=\scriptsize,
        upquote=true,
        aboveskip={1.5\baselineskip},
        columns=fixed,
        showstringspaces=false,
        extendedchars=true,
        breaklines=true,
        prebreak = \raisebox{0ex}[0ex][0ex]{\ensuremath{\hookleftarrow}},
        frame=single,
        showtabs=false,
        showspaces=false,
        showstringspaces=false,
        identifierstyle=\ttfamily,
        keywordstyle=\color[rgb]{0,0,1},
        commentstyle=\color[rgb]{0.133,0.545,0.133},
        stringstyle=\color[rgb]{0.627,0.126,0.941},
}

% Modify these paths accordingly %
\newcommand{\ConPaaSHOME}{/home/adriana/Desktop/contrail_v4/conpaas}
\newcommand{\ConPaaSCONF}{/home/adriana/Desktop/contrail_v4/conpaas}

\newenvironment{framedbox}[1]%
{\begin{framed}
 \begingroup
 \fontsize{#1}{#1}\selectfont
}
{
 \endgroup
 \end{framed}
}

\pagestyle{myheadings}
\markright{Developer's Guide: Adding a new ConPaaS service}

\begin{document}
\title{Developer's Guide: Adding a new ConPaaS service}
\date{January 17, 2012}
\thispagestyle{empty}

\begin{center}
\begingroup
\fontsize{20pt}{20pt}\selectfont
\textbf{Developer's Guide: Adding a new ConPaaS service} \linebreak
\endgroup

\begingroup
\fontsize{16pt}{16pt}\selectfont
January 17, 2012
\endgroup
\end{center}

\vspace{5 cm}

\section{Introduction}

A ConPaaS service may consist of three main entities: the manager, the agent and the frontend. The (primary) manager resides in the first VM that is started by the frontend when the service is created and its role is to manage the service by providing supporting agents, maintaining a stable configuration at any time and by permanently monitoring the service's performance. An agent resides on each of the other VMs that are started by the manager. The agent is the one that does all the work. Note that a service may contain one manager and multiple agents, or multiple managers that also act as agents.

To implement a new ConPaaS service, you must provide a new manager service, a new agent service and a new frontend service (we assume that each ConPaaS service can be mapped on the three entities architecture). To ease the process of adding a new ConPaaS service, we propose a framework which implements common functionality of the ConPaaS services. So far, the framework provides abstraction for the IaaS layer (adding support for a new cloud provider should not require modifications in any ConPaaS service implementation) and it also provides abstraction for the HTTP communication (we assume that HTTP is the preferred protocol for the communication between the three entities).   

\subsection{ConPaaS directory structure}

You can see bellow the directory structure of the ConPaaS software. The \textit{core} folder under \textit{src} contains the ConPaaS framework. Any service should make use of this code. It contains the manager hhtp server, which instantiates the python manager class that implements the required service; the agent http server that instantiates the python agent class (if the service requires agents); the iaas abstractions and other useful code. 

A new service should be added in a new python module under the \textit{ConPaaS/src/services} folder. \\

\dirtree{%
.1 ConPaaS.
.2 src.
.3 \textcolor{red}{core}.
.4 clouds.
.5 base.py.
.5 opennebula.py.
.5 ec2.py.
.4 aserver.py.
.4 aservices.py.
.4 controller.py.
.4 expose.py.
.4 http.py.
.4 iaas.py.
.4 log.py.
.4 misc.py.
.4 mserver.py.
.4 mservices.py.
.4 node.py.
.3 \textcolor{red}{services}.
.4 bagoftasks.
.4 helloworld.
.4 mapreduce.
.4 scalaris.
.4 sql.
.4 webservers.
.2 bin.
.2 config.
.2 contrib.
.2 doc.
.2 frontend.
.2 misc.
.2 sbin.
.2 scripts.
}

In the next paragraphs we describe how to add the new ConPaaS service.

\section{Service's name}

The first step in adding a new ConPaaS service is to choose a name for it. This name will be used to construct, in a standardized manner, the file names of the scripts required by this service (see below). Therefore, the names should not contain spaces, nor unaccepted characters. 

\section{Scripts}
To function properly, ConPaaS uses a series of configuration files and scripts. Some of them must be modified by the administrator, i.e. the ones concerning the cloud infrastructure, and the others are used, ideally unchanged, by the manager and/or the agent. A newly added service would ideally function with the default scripts. If, however, the default scripts are not satisfactory (for example the new service would need to start something on the VM, like a memcache server) then the developers must supply a new script/config file, that would be used instead of the default one. This new script's name must be preceded by the service's chosen name (as described above) and will be selected by the frontend at run time to generate the contextualization file for the manager VM. (If the frontend doesn't find such a script/config file for a given service, then it will use the default script). \textbf{Note that some scripts provided for a service do not replace the default ones, intead they will be concatenated to them (see below the agent and manager configuration scripts).}

Below we give an explanation of the scripts and configuration files used by a ConPaaS service (there are other configuration files used by the frontend but are not relevant to the ConPaaS service). Basically there are two scripts that a service uses to boot itself up - the manager contextualization script, which is executed after the manager VM booted, and the agent contextualization script, which is executed after the agent VM booted. These scripts are composed of several parts, some of which are customizable to the needs of the new service.      

In the ConPaaS home folder (\textdollar CONPAAS\_HOME) there is the \textit{config} folder that contains configuration files in the INI format and the \textit{scripts} folder that contains executable bash scripts. Some of these files are specific to the cloud, other to the manager and the rest to the agent. These files will be concatenated in a single contextualization script, as described below.

\begin{itemize}
\item Files specific to the Cloud:

(1) \textdollar CONPAAS\_HOME/config/cloud/\textit{cloud\_name}.cfg, where \textit{cloud\_name} refers to the clouds supported by the system (for now opennebula and ec2). So there is one such file for each cloud the system supports. These files are filled in by the administrator. They contain information such as the username and password to access the cloud, the OS image to be used with the VMs, etc. These files are used by the frontend and the manager, as both need to ask the cloud to start VMs. 

(2) \textdollar CONPAAS\_HOME/scripts/cloud/\textit{cloud\_name}, where \textit{cloud\_name} refers to the clouds supported by the system (for now opennebula and ec2). So, as above, there is one such file for each cloud the system supports. These scrips will be included in the contextualization files. For example, for opennebula, this file sets up the network. 

\item Files specific to the Manager:

(3) \textdollar CONPAAS\_HOME/scripts/manager/manager-setup, which prepares the environment by copying the ConPaaS source code on the VM, unpacking it, and setting up the PYTHONPATH environment variable.

(4) \textdollar CONPAAS\_HOME/config/manager/\textit{service\_name}-manager.cfg, which contains configuration variables specific to the service manager (in INI format). If the new service needs any other variables (like a path to a file in the source code), it should provide an annex to the default manager config file. This annex must be named \textit{service\_name}-manager.cfg and will be concatenated to default-manager.cfg

(5) \textdollar CONPAAS\_HOME/scripts/manager/\textit{service\_name}-manager-start, which starts the server manager and any other programs the service manager might use. 

(6) \textdollar CONPAAS\_HOME/sbin/manager/\textit{service\_name}-cpsmanager (will be started by the \textit{service\_name}-manager-start script), which starts the manager server, which in turn will start the requested manager service.

Scripts (1), (2), (3), (4) and (5) will be used by the frontend to generate the contextualization script for the manager VM. After this scripts executes, a configuration file containing the concatenation of (1) and (4) will be put in \textdollar ROOT\_DIR/config.cfg and then (6) is started with the config.cfg file as a parameter that will be forwarded to the new service.

Examples:

\lstinputlisting[language=bash, caption=Script (1) - ConPaaS/config/cloud/opennebula.cfg,label=lst:opennebulacfg]{\ConPaaSCONF/config/cloud/opennebula.cfg}

\lstinputlisting[language=bash, caption=Script (2) - ConPaaS/scripts/cloud/opennebula,label=lst:opennebulascr]{\ConPaaSCONF/scripts/cloud/opennebula}

\lstinputlisting[language=bash, caption=Script (3) - ConPaaS/scripts/manager/manager-setup,label=lst:managersetup]{\ConPaaSCONF/scripts/manager/manager-setup}

\lstinputlisting[language=bash, caption=Script (4) - ConPaaS/config/manager/default-manager.cfg,label=lst:managercfg]{\ConPaaSCONF/config/manager/default-manager.cfg}

\lstinputlisting[language=bash, caption=Script (5) - ConPaaS/scripts/manager/default-manager-start,label=lst:managerstart]{\ConPaaSCONF/scripts/manager/default-manager-start}

\lstinputlisting[language=python, caption=Script (6) - ConPaaS/sbin/manager/default-cpsmanager,label=lst:managersbin]{\ConPaaSHOME/sbin/manager/default-cpsmanager}

\item Files specific to the Agent

They are similar to the files described above for the manager, but this time the contextualization file is generated by the manager. 

\subsection{Scripts and config files directory structure}

Bellow you can find the directory structure of the scripts and configuration files described above.

\dirtree{%
.1 ConPaaS.
.2 \textcolor{blue}{config}.
.3 \textcolor{red}{agent}.
.4 default-manager.cfg.
.4 java-manager.cfg.
.4 php-manager.cfg.
.3 \textcolor{red}{cloud}.
.4 ec2.cfg.
.4 opennebula.cfg.
.4 clouds-template.cfg.
.3 \textcolor{red}{manager}.
.4 default-manager.cfg.
.2 \textcolor{blue}{sbin}.
.3 \textcolor{red}{agent}.
.4 default-cpsmagent.
.4 java-cpsagent.
.4 php-cpsagent.
.3 \textcolor{red}{manager}.
.4 default-cpsmanager.
.4 php-cpsmanager.
.2 \textcolor{blue}{scripts}.
.3 \textcolor{red}{agent}.
.4 agent-setup.
.4 default-agent-start.
.4 java-agent-start.
.4 php-agent-start.
.3 \textcolor{red}{cloud}.
.4 ec2.
.4 opennebula.
.3 \textcolor{red}{manager}.
.4 default-manager-start.
.4 java-manager-start.
.4 php-manager-start.
.4 manager-setup.
}

\end{itemize}

\section{Implementing a new ConPaaS service}

In this section we describe how to implement a new ConPaaS service by providing an example which can be used as a starting point. The new service is called \textit{helloworld} and will just generate helloworld strings. Thus, the manager will provide a method, called get\_helloworld which will ask all the agents to return a 'helloworld' string (or another string chosen by the manager). 

We will start by implementing the agent. We will create a class, called HelloWorldAgent, which implements the required method - get\_helloworld, and put it in \textit{conpaasservices/helloworld/agent/agent.py} (Note: make the directory structure as needed and providing empty \_\_init\_\_.py to make the directory be recognized as a module path). As you can see in Listing~\ref{lst:helloworldagent}, this class uses some functionality provided in the conpaas.core package. The conpaas.core.expose module provides a python decorator (@expose) that can be used to expose the http methods that the agent server dispatches. By using this decorator, a dictionary containing methods for http requests GET, POST or UPLOAD is filled in behind the scenes. This dictionary is used by the built-in server in the conpaas.core package to dispatch the HTTP requests. The module conpaas.core.http contains some useful methods, like HttpJsonResponse and HttpErrorResponse that are used to respond to the HTTP request dispatched to the corresponding method. In this class we also implemented a method called startup, which only changes the state of the agent. This method could be used, for example, to make some initializations in the agent. We will describe later the use of the other method, check\_agent\_process.

\lstinputlisting[language=Python, caption=conpaas/services/helloworld/agent/agent.py,label=lst:helloworldagent]{\ConPaaSHOME/src/conpaas/services/helloworld/agent/agent.py}

Let's assume that the manager wants each agent to generate a different string. The agent should be informed about the string that it has to generate. To do this, we could either implement a method inside the agent, that will receive the required string, or specify this string in the configuration file with which the agent is started. We opted for the second method just to illustrate how a service could make use of the config files and also, maybe some service agents/managers need some information before having been started.

Therefore, we will provide the \textit{helloworld-agent.cfg} file (see Listing~\ref{lst:helloworldcfg}) that will be concatenated to the default-manager.cfg file. It contains a variable (\$STRING) which will be replaced by the manager.
 
\lstinputlisting[language=bash, caption= ConPaaS/config/agent/helloworld-agent.cfg,label=lst:helloworldcfg]{\ConPaaSHOME/config/agent/helloworld-agent.cfg}

Now let's implement an http client for this new agent server. See Listing~\ref{lst:helloworldagentclient}. This client will be used by the manager as a wrapper to easily send requests to the agent. We used some useful methods from conpaas.core.http, to send json objects to the agent server. 

\lstinputlisting[language=Python, caption=conpaas/services/helloworld/agent/client.py,label=lst:helloworldagentclient]{\ConPaaSHOME/src/conpaas/services/helloworld/agent/client.py}

Next, we will implement the manager in the same manner: we will write the \textit{HelloWorldManager} class and place it in the file \textit{conpaas/services/helloworld/manager/manager.py}. To make use of the IaaS abstractions, we need to instantiate a Controller which controls all the requests to the clouds on which ConPaaS is running. Note the lines: 

\begin{lstlisting}
1: self.controller = Controller( config_parser)
2: self.controller.generate_context('helloworld')
\end{lstlisting}

The first line instantiates a Controller. The controller maintains a list of cloud objects generated from the \textit{config\_parser} file. There are several functions provided by the controller which are documented in the doxygen documentation of file \textit{controller.py}. The most important ones, which are also used in the Hello World service implementation, are: \textit{generate\_context} (which generates a template of the contextualization file); \textit{update\_context} (which takes the contextualization template and replaces the variables with the supplied values); \textit{create\_nodes} (which asks for additional nodes from the specified cloud or the default one) and \textit{delete\_nodes} (which deletes the specified nodes).

Note that the \textit{create\_nodes} function accepts as a parameter a function (in our case \textit{check\_agent\_process}) that tests if the agent process started correctly in the agent VM. If an exception is generated during the calls to this function for a given period of time, then the manager assumes that the agent process didn't start correctly and tries to start the agent process on a different agent VM.   

\lstinputlisting[language=Python, caption=conpaas/services/helloworld/manager/manager.py,label=lst:helloworldmanager]{\ConPaaSHOME/src/conpaas/services/helloworld/manager/manager.py}

We can also implement a client for the manager server (see Listing~\ref{lst:helloworldmanagerclient}). This will allow us to use the command line interface to send requests to the manager, if the frontend integration is not available.

\lstinputlisting[language=Python, caption=conpaas/services/helloworld/manager/client.py,label=lst:helloworldmanagerclient]{\ConPaaSHOME/src/conpaas/services/helloworld/manager/client.py}

The last step is to register the new service to the conpaas core. One entry must be added to files \textit{conpaas/core/mservices.py} and \textit{conpaas/core/aservices.py}, as it is indicated in Listing~\ref{lst:helloworldmservices} and Listing~\ref{lst:helloworldaservices}. Because the java and php services use the same code for the agent, there is only one entry in the agent services (\textit{aservices.py}), called \textit{web} which is used by both the webservices.

\lstinputlisting[language=Python, caption=conpaas/core/mservices.py,label=lst:helloworldmservices]{\ConPaaSHOME/src/conpaas/core/mservices.py}

\lstinputlisting[language=Python, caption=conpaas/core/aservices.py,label=lst:helloworldaservices]{\ConPaaSHOME/src/conpaas/core/aservices.py}

\section{Integrating the new service with the frontend}

So far there is no easy way to add a new frontend service. Each service may require distinct graphical elements. In this section we explain how the Hello World frontend service has been created.

\subsection{Manager states}
As you have noticed in the Hello World manager implementation, we used some standard states, e.g. INIT, ADAPTING, etc. By calling the \textit{get\_service\_info} function, the frontend knows in which state the manager is. Why do we need these standardized stated? As an example, if the manager is in the ADAPTING state, the frontend would know to draw a loading icon on the interface and keep polling the manager. 

\subsection{Files to be modified}

\dirtree{%
.1 frontend.
.2 www.
.3 \textcolor{blue}{create.php}.
.3 lib.
.4 service.
.5 factory.
.6 \textcolor{blue}{\_\_init\_\_.php}.
}

Several lines of code must be added to the two files above for the new service to be recognized. If you look inside these files, you'll see that knowing where to add the lines and what lines to add is self-explanatory. 

\subsection{Files to be added}

\dirtree{%
.1 frontend.
.2 www.
.3 \textcolor{red}{helloworld.php}.
.2 lib.
.3 service.
.4 helloworld.
.5 \textcolor{red}{\_\_init\_\_.php}.
.3 ui.
.4 instance.
.5 helloworld.
.6 \textcolor{red}{\_\_init\_\_.php}.
.2 images.
.3 \textcolor{red}{helloworld.png}.
}

\end{document}

