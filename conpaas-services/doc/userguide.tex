\documentclass[10pt]{article}
\usepackage{listings}
\usepackage{framed}
\usepackage{hyperref}

\newenvironment{what}
{\begin{description} \item [What is happening now?] \hfill \\}
{\end{description}}

\newenvironment{framedbox}[1]%
{\begin{framed}
 \begingroup
 \fontsize{#1}{#1}\selectfont
}
{
 \endgroup
 \end{framed}
}


\begin{document}

\title{ConPaaS User Manual}
\author{Ismail El Helw \and Guillaume Pierre}
\date{Version 1.0.0}
\maketitle

\vfil
\tableofcontents
\newpage

\section{Introduction}

ConPaaS is an open-source runtime environment for hosting applications
in the cloud which aims at offering the full power of the cloud to
application developers while shielding them from the associated
complexity of the cloud.

ConPaaS is designed to host both high-performance scientific
applications and online Web applications. It runs on a variety of
public and private clouds, and is easily extensible.  ConPaaS
automates the entire life-cycle of an application, including
collaborative development, deployment, performance monitoring, and
automatic scaling. This allows developers to focus their attention on
application-specific concerns rather than on cloud-specific details.

ConPaaS is organized as a collection of \textbf{services}, where each
service acts as a replacement for a commonly used runtime environment.
For example, to replace a MySQL database, ConPaaS provides a
cloud-based MySQL service which acts as a high-level database
abstraction. The service uses real MySQL databases internally, and
therefore makes it easy to port an cloud application to ConPaaS.
Unlike a regular centralized database, however, it is self-managed and
fully elastic: one can dynamically increase or decrease its processing
capacity by requesting it to reconfigure itself with a different
number of virtual machines.

ConPaaS currently contains six services: 

\begin{itemize}
\item \textbf{Two Web hosting services} respectively specialized for
  hosting PHP and JSP applications;
\item \textbf{MySQL database} service;
\item \textbf{Scalarix service} offering a scalable in-memory
  key-value store;
\item \textbf{MapReduce service} providing thewell-known
  high-performance computation framework;
\item \textbf{TaskFarming service} high-performance batch processing.
\end{itemize}

ConPaaS applications can be composed of any number of services. For
example, a bio-informatics application may make use of a PHP and a
MySQL service to host a Web-based frontend, and link this frontend to
a MapReduce backend service for conducting high-performance genomic
computations on demand.

\section{ConPaaS usage overview}

Most operations in ConPaaS can be done using the ConPaaS frontend,
which gives a Web-based interface to the system. The front-end allows
users to register, create services, upload code and data to the
services, and configure each service. 

\begin{itemize}
\item The Dashboard page displays the list of services currently
  active in the system. Beware: each active service uses credits, even
  if it is in ``stopped'' state. To stop using credits you must
  terminate the services completely.
\item Each service comes with a separate page which allows one to
  configure it, upload code and data, and scale it up and down.
\end{itemize}

All the functionalities of the frontend are also available using a
command-line interface. This allows one to script commands for
ConPaaS. The command-line interface also features additional advanced
functionalities, which are not available using the front-end.

\section{The PHP Web hosting service}

The PHP Web hosting service is dedicated to hosting Web applications
written in PHP. It can also host static Web content.

\subsection{Basic usage with the front-end}

\begin{description}
\item[Create a service.] Click on ``create new service'', then select
  ``PHP service''. This operation starts a new ``Manager'' virtual
  machine. The manager is in charge of taking care of the service, but
  it does not host Web applications itself. To make use of the PHP
  service you need to start it.
\item[Start a service.] Click on ``start'', this will create a new
  virtual machine which can host your Web application.
\item[Rename the service.] By default all new PHP services are named
  ``New PHP service.'' To give it a meaningful name, click on this
  name in the service-specific page and enter a new name.
\item[Upload an application to the service.] Applications can be
  uploaded in the form of a \texttt{tar} or \texttt{zip} archive.
  Attention: the archive must expand \emph{in the current directory}
  rather than in a subdirectory. The service does not immediately use
  new applications when they are uploaded. The frontend shows the list
  of versions that have been uploaded; choose one version and click
  ``make active'' to activate it. Note that the frontend only allows
  uploading archives smaller than a certain size. To upload large
  archives, you must use the command-line tool (see Advanced usage).
\item[Access your application.] The frontend gives a link to the
  running application. This URL will remain valid as long as you do
  not stop the service.
\item[Scale the service up and down.] The front-end shows the list of
  virtual machine instances used to run a service. The minimum
  configuration to run a PHP application requires two instances: one
  manager instance, and one instance containing a load balancer, a web
  server and a PHP server. To scale the capacity up or down, click the
  numbers below to request adding or removing servers. The system
  reconfigures itself without any service interruption.
\item[Stop the service.] When you do not need to run the application
  any more, click ``stop'' to stop the service. This stops all
  instances except the manager which keeps on running. Beware that the
  manager still uses credit while it is running. To stop using credit
  you must terminate the service.
\item[Terminate the service.] Click ``terminate'' to terminate the
  service. At this point all the state of the service manager will be
  lost.
\end{description}

\subsection{Advanced usage with the command-line tool}

The PHP service can also be controled using the \texttt{cpsclient.web}
script. This command-line tool can issue the same operations as the
front-end, except creating a new service. It also has additional
functionalities which are useful for advanced usage.

To use the \texttt{cpsclient.web} tool you must first set the
PYTHONPATH variable to contain the full path name of the ``src''
subdirectory within the ConPaaS directory:
\begin{verbatim}
$ export PYTHONPATH=/path/to/conpaas/src
\end{verbatim}

The \texttt{cpsclient.web} tool always takes the URL of the service
manager as its first argument. This URL is provided by the front-end.

\vspace{1em}

\begin{description}
\item[List all options of the command-line tool.]~
\begin{verbatim}
$ ./cpsclient.web help
\end{verbatim}

\item[Start the service.]~
\begin{verbatim}
$ ./cpsclient.web http://x-x-x-x/ startup
\end{verbatim}

\item[Stop the service.]~
\begin{verbatim}
$ ./cpsclient.web http://x-x-x-x/ shutdown
\end{verbatim}

\item[Upload a new version of the application]~
\begin{verbatim}
$ ./cpsclient.web http://x-x-x-x/ upload_code_version path/to/archive.zip
\end{verbatim}
Note that this operation allows one to upload bigger archive files than the frontend.

\item[Scale the service up and down.]~
\begin{verbatim}
$ ./cpsclient.web http://x-x-x-x/ add_nodes -h
Usage: add_nodes

Options:
  -h, --help            show this help message and exit
  -p PROXY, --proxy=PROXY
  -w WEB, --web=WEB     
  -b BACKEND, --backend=BACKEND

$ ./cpsclient.web http://x-x-x-x/ add_nodes -w 1 -b 1

$ ./cpsclient.web http://x-x-x-x/ remove_nodes -h
Usage: remove_nodes

Options:
  -h, --help            show this help message and exit
  -p PROXY, --proxy=PROXY
  -w WEB, --web=WEB     
  -b BACKEND, --backend=BACKEND

$ ./cpsclient.web http://x-x-x-x/ remove_nodes -w 1 -b 1
\end{verbatim}

\item[Set the service in debug mode.] By default the PHP service does
  not display anything in case PHP errors occur while executing the
  application. This setting is useful for production, when you do not
  want to reveal internal information to external users. While
  developing an application it is however useful to let PHP display
  erors.
\begin{verbatim}
$ ./cpsclient.web http://x-x-x-x/ update_php_configuration \
  -p "display_errors=On" 
\end{verbatim}
\end{description}


\section{The Java Web hosting service}

The Java Web hosting service is dedicated to hosting Web applications
written in Java using JSP or servlets. It can also host static Web
content.

\subsection{Basic usage with the front-end}

\begin{description}
\item[Create a service.] Click on ``create new service'', then select
  ``Java service''. This operation starts a new ``Manager'' virtual
  machine. The manager is in charge of taking care of the service, but
  it does not host Web applications itself. To make use of the Java
  service you need to start it.
\item[Start a service.] Click on ``start'', this will create a new
  virtual machine which can host your Web application.
\item[Rename the service.] By default all new Java services are named
  ``New Java service.'' To give it a meaningful name, click on this
  name in the service-specific page and enter a new name.
\item[Upload an application to the service.] Applications can be
  uploaded in the form of a \texttt{war} file. The service does not
  immediately use new applications when they are uploaded. The
  frontend shows the list of versions that have been uploaded; choose
  one version and click ``make active'' to activate it. Note that the
  frontend only allows uploading archives smaller than a certain size.
  To upload large archives, you must use the command-line tool (see
  Advanced usage).
\item[Access your application.] The frontend gives a link to the
  running application. This URL will remain valid as long as you do
  not stop the service.
\item[Scale the service up and down.] The front-end shows the list of
  virtual machine instances used to run a service. The minimum
  configuration to run a Java application requires two instances: one
  manager instance, and one instance containing a load balancer, a web
  server and a Java server. To scale the capacity up or down, click the
  numbers below to request adding or removing servers. The system
  reconfigures itself without any service interruption.
\item[Stop the service.] When you do not need to run the application
  any more, click ``stop'' to stop the service. This stops all
  instances except the manager which keeps on running. Beware that the
  manager still uses credit while it is running. To stop using credit
  you must terminate the service.
\item[Terminate the service.] Click ``terminate'' to terminate the
  service. At this point all the state of the service manager will be
  lost.
\end{description}

\subsection{Advanced usage with the command-line tool}

The Java service can also be controled using the \texttt{cpsclient.web}
script. This command-line tool can issue the same operations as the
front-end, except creating a new service. It also has additional
functionalities which are useful for advanced usage.

To use the \texttt{cpsclient.web} tool you must first set the
PYTHONPATH variable to contain the full path name of the ``src''
subdirectory within the ConPaaS directory:
\begin{verbatim}
$ export PYTHONPATH=/path/to/conpaas/src
\end{verbatim}

The \texttt{cpsclient.web} tool always takes the URL of the service
manager as its first argument. This URL is provided by the front-end.

\vspace{1em}

\begin{description}
\item[List all options of the command-line tool.]~
\begin{verbatim}
$ ./cpsclient.web help
\end{verbatim}

\item[Start the service.]~
\begin{verbatim}
$ ./cpsclient.web http://x-x-x-x/ startup
\end{verbatim}

\item[Stop the service.]~
\begin{verbatim}
$ ./cpsclient.web http://x-x-x-x/ shutdown
\end{verbatim}

\item[Upload a new version of the application]~
\begin{verbatim}
$ ./cpsclient.web http://x-x-x-x/ upload_code_version path/to/archive.war
\end{verbatim}
Note that this operation allows one to upload bigger archive files than the frontend.

\item[Scale the service up and down.]~
\begin{verbatim}
$ ./cpsclient.web http://x-x-x-x/ add_nodes -h
Usage: add_nodes

Options:
  -h, --help            show this help message and exit
  -p PROXY, --proxy=PROXY
  -w WEB, --web=WEB     
  -b BACKEND, --backend=BACKEND

$ ./cpsclient.web http://x-x-x-x/ add_nodes -w 1 -b 1

$ ./cpsclient.web http://x-x-x-x/ remove_nodes -h
Usage: remove_nodes

Options:
  -h, --help            show this help message and exit
  -p PROXY, --proxy=PROXY
  -w WEB, --web=WEB     
  -b BACKEND, --backend=BACKEND

$ ./cpsclient.web http://x-x-x-x/ remove_nodes -w 1 -b 1
\end{verbatim}
\end{description}


\section{The MySQL database service}

\section{The Scalarix key-value store service}

\section{The MapReduce service}

\section{The TaskFarming service}

\section{Building new types of services}

The architecture of ConPaaS allows developers to build new types of
services. To learn how to do this, please check the ``Internals''
ConPaaS documentation.

\section{About this document}

\begin{verbatim}
Copyright (c) 2010-2012, Contrail consortium.
All rights reserved.

Redistribution and use in source and binary forms, 
with or without modification, are permitted provided
that the following conditions are met:

 1. Redistributions of source code must retain the
    above copyright notice, this list of conditions
    and the following disclaimer.
 2. Redistributions in binary form must reproduce
    the above copyright notice, this list of 
    conditions and the following disclaimer in the
    documentation and/or other materials provided
    with the distribution.
 3. Neither the name of the Contrail consortium nor the
    names of its contributors may be used to endorse
    or promote products derived from this software 
    without specific prior written permission.

THIS SOFTWARE IS PROVIDED BY THE COPYRIGHT HOLDERS AND
CONTRIBUTORS "AS IS" AND ANY EXPRESS OR IMPLIED WARRANTIES,
INCLUDING, BUT NOT LIMITED TO, THE IMPLIED WARRANTIES OF
MERCHANTABILITY AND FITNESS FOR A PARTICULAR PURPOSE ARE
DISCLAIMED. IN NO EVENT SHALL THE COPYRIGHT HOLDER OR
CONTRIBUTORS BE LIABLE FOR ANY DIRECT, INDIRECT, INCIDENTAL,
SPECIAL, EXEMPLARY, OR CONSEQUENTIAL DAMAGES (INCLUDING, 
BUT NOT LIMITED TO, PROCUREMENT OF SUBSTITUTE GOODS OR 
SERVICES; LOSS OF USE, DATA, OR PROFITS; OR BUSINESS 
INTERRUPTION) HOWEVER CAUSED AND ON ANY THEORY OF LIABILITY,
WHETHER IN CONTRACT, STRICT LIABILITY, OR TORT
(INCLUDING NEGLIGENCE OR OTHERWISE) ARISING IN ANY WAY OUT
OF THE USE OF THIS SOFTWARE, EVEN IF ADVISED OF THE
POSSIBILITY OF SUCH DAMAGE.
\end{verbatim}

\end{document}

