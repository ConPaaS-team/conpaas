\documentclass[10pt]{article}

\usepackage{listings}
\usepackage{framed}
\usepackage{hyperref}
\usepackage{url}

% \newenvironment{what}
% {\begin{description} \item [What is happening now?] \hfill \\}
% {\end{description}}

\newenvironment{framedbox}[1]%
{\begin{framed}
 \begingroup
 \fontsize{#1}{#1}\selectfont
}
{
 \endgroup
 \end{framed}
}


\begin{document}
\title{ConPaaS -- Administrator guide}
\author{Ismail El Helw \and Guillaume Pierre}
\maketitle

\vfil
\tableofcontents
\vfil
\newpage

\section{Creating a ConPaaS image for Amazon EC2}

The Web Hosting Service is capable of running over the Elastic Compute
Cloud (EC2) of Amazon Web Services (AWS). This section describes the
process of configuring an AWS account to run the Web Hosting Service.

\subsection{Create an EBS backed AMI on Amazon EC2}

The Web Hosting Service requires the creation of an Amazon Machine
Image (AMI) to contain the dependencies of it's processes.  The
easiest method of creating a new Elastic Block Store backed Amazon
Machine Image is to start from an already existing one, customize it
and save the resulting filesystem as a new AMI. The following steps
explains how to setup an AMI using this methodology.

\begin{enumerate}
\item Search the public AMIs for a Debian squeeze EBS AMI and run an
  instance of it.

\item Download the \verb+web-servers/conpaas_web_deps+ script and run
  it inside the instance. This script will install all of the
  dependencies of the manager and agent processes as well as create
  the necessary directory structure.

\item Clean the filesystem by removing any temporary files you may
  have created.

\item Go to the EC2 administration page at the AWS website, right
  click on the running instance and click on ``\emph{Create Image (EBS
    AMI)}''.  AWS documentation is available at
  \url{http://docs.amazonwebservices.com/AWSEC2/latest/UserGuide/index.html?Tutorial_CreateImage.html}.

\item Stop the instance.
\end{enumerate}

\subsection{Create a Security Group}

An AWS security group is an abstraction of a set of firewall rules to
limit inbound traffic. The default policy of a new group is to deny
all inbound traffic. Therefore, one needs to specify a whitelist of
protocols and destination ports that are accesible from the outside.
The Web Hosting Service uses TCP ports 80, 8080 and 9000. All three
ports should be open for all running instances.  AWS documentation is
available at
\url{http://docs.amazonwebservices.com/AWSEC2/latest/UserGuide/index.html?using-network-security.html}.

\section{Creating a ConPaaS image for OpenNebula}

The Web Hosting Service is capable of running over an OpenNebula
installation. This section describes the process of configuring an AWS
account to run the Web Hosting Service.

\emph{To be written\ldots}

\section{Setup ConPaaS Front-end}

The ConPaaS Front-end is a web application that allows users to
manager their ConPaaS services. Users can create, configure and
terminate services through it. This section describes the process of
setting up a ConPaaS frontend.

\subsection{Create a MySQL Database}

The ConPaaS frontend uses a MySQL database to store data about users
and their services. The script located in
\verb+frontend/scripts/frontend-db.sql+ creates a new user
\verb+DB_USER+ with password \verb+DB_PASSWD+ and a database
\verb+DB_NAME+. It grants all access permissions to user
\verb+DB_USER+ on the new database. Finally, it creates the database
schema. You must update the first four lines to change \verb+DB_USER+,
\verb+DB_PASSWD+ and \verb+DB_NAME+ to reasonable values.

Install a MysQL database if you don't have one already. You can now
create the database schema using this command, replacing \verb+ADMIN+
and \verb+ADMINPASSWORD+ with the MySQL administrator's name and
password:

\begin{framedbox}{8pt}\begin{verbatim}
mysql -u ADMIN -p ADMINPASSWORD  < frontend-db.sql
\end{verbatim}\end{framedbox}



\subsection{Configure the Front-end}

The ConPaaS Front-end code is a collection of PHP scripts. It can run
on any PHP-enabled Web server. The following instructions detail the
configuration of the Front-end.

\begin{enumerate}
\item Copy all files from the \verb+frontend/conf+ directory to a
  location \emph{outside} of the Web server's document root. A good
  location could be for example \verb+/etc/conpaas+. Edit each of
  these files to setup the required configuration parameters. Each
  variable should be described in the config file itself.

\item Place the PHP code found in directory \verb+frontend/www+ at the
  document root of the target web server such that the file named
  \verb+__init__.php+ is directly underneath it.

\item Edit the \verb+CONF_DIR+ variable in \verb+__init__.php+ such
  that it points to the configuration directory path.

\item Download the AWS sdk for PHP from
  \url{http://aws.amazon.com/sdkforphp/}.  Extract the sdk directory
  and rename it to \verb+aws-sdk+. Place it under the lib directory of
  the front-end source code such that \verb+lib/aws-sdk/+ contains a
  file named \verb+config-sample.inc.php+ (among others).

\item Copy \verb+lib/aws-sdk/config-sample.inc.php+ to
  \verb+lib/aws-sdk/config.inc.php+ and fill in \verb+AWS_KEY+,
  \verb+AWS_SECRET_KEY+, \verb+AWS_ACCOUNT_ID+ and
  \verb+AWS_CANONICAL_ID+ as instructed in the file's documentation.

\item Make sure that the Web server's document directory contains a
  subdirectory named \verb+code+ and containing the following files:
  \verb+agent-start+, \verb+agent-stop+, \verb+ConPaaSWeb.tar.gz+,
  \verb+ec2-agent-user-data+, \verb+ec2-manager-user-data+, and
  \verb+manager-start+. These files contain the entire implementation
  of the Web hosting service. They are downloaded by newly created VM
  instances upon startup. Make sure that variable \verb+SOURCE+ from
  the frontend's configuration file \verb+manager-user-data+
  points to the URL of this directory.
\end{enumerate}

At this point, your front-end should be working!

\section{Miscellaneous}
\subsection{The credit system}

The frontend is designed to maintain accounting of resources used by
each user. When a new user is created, (s)he receives a number of
credits as specified in the ``main.ini'' configuration file. Later on,
one credit is substracted each time a VM is executed for (a fraction
of) one hour. The administrator can change the number of credits by
directly editing the frontend's database. 

\subsection{Application sandboxing}

The default ConPaaS configuration creates strong snadboxing so that
applications cannot open sockets, access the file system, execute
commands, etc. This makes the platform relatively secure against
malicious applications. On the other hand, it strongly restricts the
actions that ConPaaS applications can do. To reduce these security
measures to a more usable level, you need to edit two files:

\begin{itemize}
\item To change restrictions applied to PHP applications, edit file
  \verb+web-servers/etc/fpm.tmpl+ to change the list of
  \verb+disable\_functions+. Do not forget to recreate a file
  \verb+ConPaaSWeb.tar.gz+ out of the entire \verb+web-servers+
  directory, and to copy it at the URL specified in file
  \verb+frontend/conf/manager-user-data+.
\item To change restrictions applied to Java applications, edit file
  ``web-servers/etc/tomcat-catalina.policy''. Do not forget to
  recreate a file ConPaaSWeb.tar.gz out of the entire ``web-servers''
  directory, and to copy it at the URL specified in file
  ``frontend/conf/manager-user-data''.
\end{itemize}


\end{document}
